\documentclass [letterpaper,11pt]{article}
\usepackage{fullpage,amsmath,hyperref}

\newcommand{\urlnofont}[1]{\urlstyle{same}\url{#1}}

\begin{document}

\begin{center}
\Large COMP 172 --- Discrete Structures --- Fall 2017
\\ \normalsize CRN 18282
\end{center}

\noindent\begin{tabular}{@{}ll}
\textbf{Instructor:} & Phillip Kirlin \\
\textbf{Meetings:} & MWF, 1--1:50pm, Briggs 119 \\
\textbf{Course website:} & \texttt{http://www.cs.rhodes.edu/172}\\
\textbf{Email:} &\texttt{kirlinp@rhodes.edu} (please include ``CS 172'' somewhere in the subject)\\
\textbf{Office hours:} & See website for scheduled office hours; I am also available by appointment.

\end{tabular}
\begin{description}

\item[Course Overview:] An introduction to and survey of the mathematics used in computer science including propositional and predicate logic, proof techniques, induction, 
sets, counting, probability, functions, and relations. Other topics may be included as time permits.

\item[Text:] Susanna S.~Epp.  \textit{Discrete Mathemematics: An Introduction to 
Mathematical Reasoning.}  Brief edition.  2011.  

You may also use the 4th edition of Epp's book, \textit{Discrete Mathematics with Applications}, as this book covers all the same material as our ``brief edition'' does.  You may use an earlier
edition at your own risk.

\item[Prerequisites:] COMP 141 (Computer Science I).  Additionally, this class assumes knowledge of 
mathematics through pre-calculus, and a high level of mathematical maturity.

\item[Coursework:] \

\begin{tabular}{lcc} 
& Tentative weight & Tentative date \\ \hline
Problem sets & 35\% & \\
Quizzes & 10\% & \\
Midterm 1 & 15\% & Monday, October 9, 6pm \\
Midterm 2 & 15\% & Thursday, November 9, 6pm \\
Comprehensive final exam & 25\% & Saturday, December 9, 1pm \\
\end{tabular}

Final letter grades of A--, B--, C--, and D-- are guaranteed with final course grades of 90\%, 80\%,
70\%, and 60\%, respectively.
If your final course grade falls near a letter grade boundary,
I may take into account class participation, attendance, and/or improvement during the semester.

Any written assignments are due at the beginning of class on the assigned date.
  In general, late work
will not be accepted, so turn in whatever you have completed by the due date.
Homework assignments should be written neatly. Poorly written work will not be graded. 
All pages of assignments should be stapled together.



If you have a valid reason for a makeup exam, inform your instructor
   as soon as you know.  A valid reason is a medical emergency, a death in the family, 
   religious observation, a college-sponsored off-campus activity, and, quite frankly, 
   very little else.  Generally, assignment extensions will only be granted for 
   \emph{unplanned} circumstances (e.g., the first two reasons above). 

\item[Office Hours:]
In addition to regular office hours, am also available immediately after class for 
short questions.  You never need an appointment to see me during regular office hours; you
can just come by.  Outside of regular office hours, feel free to stop by my office,
and if I have time, I'll try to help you.  If I don't have time at that moment, we'll set up an
appointment for a different time.
Don't be shy about coming by my office or sending me email
if you can't make my regular office hours.  I always set aside time each week for ``unscheduled'' office hours.

\item[Workload:]
It is important to stay current with the material.  Most of this course deals with
mathematical proofs, a topic which often gives people trouble at first. You should be prepared to devote  at least 2--3 hours outside of class for each in-class lecture.  In particular, you should expect to spend a significant amount of time for this course working on practice problems and homework assignments.	  Do not wait to the last minute to start your homework.




\item[Class Conduct:] \
   \begin{itemize}\setlength{\itemsep}{0em}\setlength{\parskip}{0pt}
   	\item I encourage everyone to participate in class.  Raise your hand if you have
	a question or comment.  Please don't be shy about this; if you are confused about
	something, it is likely that someone else is confused as well.
		Teaching and learning is a partnership between the instructor and the students, and asking questions not only helps you understand the material, it also
		helps me know what I'm doing right or wrong.
			     \item Do not use your cell phone while in class, and 
			     keep the ringer on silent.
     \item  If you cannot make it to class for whatever reason, make sure that
       you know what happened during the lecture that you missed. It is
       your responsibility, and nobody else's, to do so.  The best way to do this is
       to ask a classmate.  
     \item  If you have to leave a class early, inform the instructor in
       advance. It is rude to walk out in the middle of a
       lecture. 
     \end{itemize}

   
\item[Students With Disabilities:]
If you have a documented disability and wish to receive academic accommodations, please
contact the Office of Student Disability Services at x3885 as soon as possible.
  
\item[Collaboration:]
Students should talk to each other about the subject matter of this class and help each other.  It is fine to discuss the readings, lectures, and problems and ask questions about them. I encourage such questions in class as well as elsewhere. However, there is a line past which you must not go, e.g., copying a solution from a fellow student, book, website, etc. will cause you to fail the course, or worse. If a significant part of one of your solutions is due to someone else, or something you've read, then you must acknowledge your source. Failure to do so is a serious academic violation. Of course, even after you acknowledge your source you must still understand your solution and write it in your own words. Copying a solution from the web, a book, or classmate will result in failure even if you acknowledge your source, unless you put it in quotation marks and say something like, ``Here is Amy's solution, but I don't understand it enough to absorb it and write it in my own words.'' However, this won't get you much --- if any --- credit. 

A good rule of thumb to follow is to not write up any solutions to homework assignments
in a group.  Feel free to discuss things with your classmates, but you should write up
your solutions alone.   
  
\item[Academic Integrity:]
   Plagiarism, cheating, and similar anti-intellectual behavior are serious violations of academic ethics and will be correspondingly penalized. If you are concerned about a possible violation of this kind, please talk with me.  I understand that being a student at Rhodes 
can be stressful sometimes and you will have many demands on your time.  However, I would
much rather have you turn in a partially-completed assignment or do poorly on a test than
have you violate the Rhodes Honor Code.  I can --- and very much want to --- help you if you don't understand the material, but violations of academic integrity will be dealt with harshly.
   
Unless otherwise specified, everything you submit in this course must be your own work and represent
your individual effort.  These are all included in the definition of reportable Honor Code violations for this course:
copying all or part of a solution to a problem, downloading a solution from the internet and submitting it as your own, having someone else provide the solution for you, or allowing someone else to copy from you.	If you have any doubt about what type of behavior is acceptable, please talk with me.

\item[Mutual Respect:]
Rhodes College is committed to creating an academic climate that is safe, respectful, and appreciative of all students, staff, and faculty regardless of race, ethnicity, sexual orientation, gender identity, age, size, socioeconomic background, religion, spirituality, physical ability, mental ability, or any other aspect of one's identity. � A climate of mutual respect allows us to ask difficult questions and to participate in honest discussions, even in the context of strong disagreement. Creating this kind of open, honest, and respectful climate is our mutual responsibility. 


\item[Course Topics:]\
	\begin{itemize}
		\item Propositional and Predicate Logic (ch.~1--3, approx.~4 weeks)
		\item Elementary Number Theory (ch.~4, approx.~2 weeks)
		\item Summations, Recurrences and Mathematical Induction (ch.~5, approx.~2 weeks)
		\item Set Theory (ch.~6, approx.~1 1/2 weeks)
		\item Functions (ch.~7, approx.~1 1/2 weeks)
		\item Counting and Probability (ch.~9, approx.~2 weeks)
		\item Relations (ch.~8, approx.~1 week)
		%\item Graph Theory (ch.~10, if time permits)

\end{itemize}
\end{description}
   \end{document}
