\documentclass{article}
\usepackage{amsmath,fullpage,amssymb,parskip,etoolbox}

\newcommand{\nott}{{\sim}}
\newcommand{\Z}{\mathbb{Z}}
\newcommand{\R}{\mathbb{R}}
\newcommand{\Q}{\mathbb{Q}}
\newcommand{\pset}[1]{\mathcal P \left({#1}\right)}
\newcommand{\indentpar}{\addtolength{\leftskip}{5mm}}
\newcommand{\stopindentpar}{\addtolength{\leftskip}{-5mm}}
\newcommand{\tautology}{\textbf{t}}
\newcommand{\contradiction}{\textbf{c}}

\setlength{\parindent}{0em}

\newtoggle{sol}
\newtoggle{notes}

\toggletrue{sol}
\togglefalse{notes}

\iftoggle{notes}{
	\newcommand{\proofnote}[1]{[\textit{Note: #1}]}
}{
	\newcommand{\proofnote}[1]{}
}

\begin{document}

\section{Chapter 4: Number Theory}



\subsection{Even and Odd}

The sum of any two even integers is even.

The product of any two even integers is even.

\subsection{Rational}

\subsection{Divisibility}

\subsection{Quotient-Remainder Theorem}

%%%%%%%%%%%%%%%%%%%%%%%%%%%%%%%%%%%%%%%%%%%%%%%%%%%%%%%%%%%%%%%%%%%%%%%%%%%%%%%%%%%%%%%%%%%%%%%%%%%%%

The square of any integer can be written as $3s$ or $3s+1$ for some integer $s$.

\iftoggle{sol}{

In symbols: $\forall n \in \Z \ \exists s \in \Z \ (n^2 = 3s) \lor (n^2=3s+1)$

\textbf{Proof:}

Suppose $n$ is an arbitrary integer.

\proofnote{We must show that we can write $n^2$ as either $3(\textrm{int})$ or $3(\textrm{int})+1$.  However, we don't
have any information about $n$ other than being an integer, so we're stuck.  We will use the QRT to get unstuck.}

By the quotient-remainder theorem, there exists an integer $q$ such that $n=3q$ or $n=3q+1$ or $n=3q+2$.

Case 1: Assume $n=3q$. 
\begin{align*}
n^2 &= (3q)^2 & \text{by substitution} \\
&= 3(3q^2) & \text{by algebra}
\end{align*}
\indentpar
Let $s =3q^2$.  $s$ is an integer by closure of the integers under multiplication and addition.

Therefore, $n^2=3s$.

\stopindentpar

Case 2: Assume $n=3q+1$.
\begin{align*}
n^2 &= (3q+1)^2 & \text{by substitution} \\
&= 9q^2 + 6q + 1 & \text{by algebra}\\
&= 3(3q^2 +2q) + 1 & \text{by algebra}
\end{align*}
\indentpar
Let $s =3q^2+2q$.  $s$ is an integer by closure of the integers under multiplication and addition.

Therefore, $n^2=3s+1$.

\stopindentpar

Case 3: Assume $n=3q+2$.
\begin{align*}
n^2 &= (3q+2)^2 & \text{by substitution} \\
&= 9q^2 + 12q + 4 & \text{by algebra}\\
&= 9q^2 + 12q + 3+1 & \text{by algebra}\\
&= 3(3q^2 +4q+1) + 1 & \text{by algebra}
\end{align*}
\indentpar
Let $s =3q^2+4q+1$.  $s$ is an integer by closure of the integers under multiplication and addition.

Therefore, $n^2=3s+1$.

\stopindentpar

The QRT tells us that one of these three cases must apply, and in each case, we have proved that $n^2=3s$ or $n^2=3s+1$
for some integer $s$.
}

%%%%%%%%%%%%%%%%%%%%%%%%%%%%%%%%%%%%%%%%%%%%%%%%%%%%%%%%%%%%%%%%%%%%%%%%%%%%%%%%%%%%%%%%%%%%%%%%%%%%%



\subsection{Proof by Contradiction and Contraposition}


\end{document}