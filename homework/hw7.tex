\documentclass[12pt, letterpaper]{report}
\usepackage{amsmath,fullpage,graphicx,amssymb}

\setlength{\parindent}{0em}

\newcommand{\nott}{{\sim}}
\newcommand{\Z}{\mathbb{Z}}

\begin{document}

{\textbf{Discrete Structures, Fall 2023, Homework 7}}

\medbreak

\textit{You must write the solutions to these problems legibly on your own paper, with
the problems in sequential order, and with all sheets stapled together.}

\bigskip

Prove each of the following statements using ``regular'' or ``weak'' induction.

\begin{enumerate}

       
       \item $\forall n \in \Z^{\geq 0} \ \displaystyle \sum_{i=0}^n (3i^2-i) = n^2(n+1)$      

       \item Prove $\displaystyle \forall n \in \Z^+ \ \prod_{i=1}^n i(i+1) = (n+1)(n!)^2$

Hint: Recall that $n!$ denotes \emph{the factorial of $n$} or \emph{$n$ factorial}, and is defined as $n!=n(n-1)(n-2)\cdots2\cdot 1$, with $0!$ defined to be 1.  
However, an alternate
formula involving recursion is the following:
$$n! = \begin{cases} 1 &\text{ if } n= 0 \\
n \cdot (n-1)! &\text{ otherwise}\end{cases}$$
This recursive definition will be useful during the inductive step.     

\item $\forall n \in \Z^{\geq 0} \ n(n+1)$ is even.

Note: This is the same problem as question 1 on the last homework (prove that the product of any two consecutive integers is even). 
On that homework, you did this with the quotient-remainder theorem.  On this homework, you should use induction (do not use the QRT here).

\item $\forall n \in \Z^{\geq 0} \ \ 5 \mid 7^n - 2^n$.   

Hint: Remember that $7^{k+1} = 7\cdot 7^k$ and $2^{k+1} = 2\cdot 2^k$.  You can complete the inductive step of this 
proof in two different ways.  The easier way involves manipulating the inductive hypothesis to get either $7^k$ or $2^k$ alone
on one side of the equals sign, then substituting that into a piece of an equation in the inductive step.

\end{enumerate}

 



\end{document}
