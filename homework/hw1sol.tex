\documentclass[12pt, letterpaper]{report}
\usepackage{amsmath,fullpage,graphicx}

\setlength{\parindent}{0em}
\newcommand{\nott}{{\sim}}
\newcommand{\Z}{\mathbb{Z}}
\newcommand{\powerset}[1]{\mathcal P \left({#1}\right)}
\newcommand{\proofnote}[1]{[\textit{Note: #1}]}
\newcommand{\ans}{\textbf{Answer: }}


\begin{document}

{\textbf{Discrete Structures, Fall 2017, Homework 1 Solutions}}

\medbreak


You must write the solutions to these problems legibly on your own paper, with
the problems in sequential order, and with all sheets stapled together.

\medbreak

\begin{enumerate}

	\item Convert the following sentences to logical statements using symbols assuming that ``$p$'', ``$b$''
	and ``$m$'' represent the propositions below.  
	
	$p$ = ``Morgan is taking a physics class.'' \\
	$b$ = ``Morgan is taking a biology class.'' \\
	$m$ = ``Morgan is taking a math class.''
	
	\begin{enumerate}
		\item Morgan is taking a biology class and a math class, but not a physics class.
		
		\ans $b \land m \land \nott p$.
		
		\item Morgan is taking a physics class, and either a biology or math class (but not both bio and math).
		
		\ans $p \land (b \lor m) \land (\nott b \lor \nott m)$
		\\ OR \\
		\ans $p \land [(b \land \nott m) \lor (\nott b \land m)]$
		
		\item Morgan is taking a physics class, and either a biology or math class (perhaps both bio and math).
		
		\ans $p \land (b \lor m)$.
		
		\item If Morgan is taking a biology class, then they are also taking a math class, but if Morgan is not taking a 
		biology class, then they are taking a physics class.
		
		\ans $(b \to m) \land (\nott b \to p)$
	\end{enumerate}

	\item For each of the sentences below, determine if the sentence is a statement.  If
	the sentence is a statement, tell whether it is true or false.
	\begin{enumerate}
		\item If a tree falls in the forest and no one is around to hear it, does it make a sound?
		\\ \ans Not a statement; it's a question.
		\item If $2+2=5$, then I am the very model of a modern major-general.
		\\ \ans True statement (left side of this implication is clearly false, which means the whole statement is true).
		\item This statement refers to itself.
		\\ \ans True statement.
		\item This statement is false.
		\\ \ans Not a statement (it's a paradox).
	\end{enumerate}
	
	\item Express the negations of the following statements in normal English sentences.
	\begin{enumerate}
		\item Sally is a computer science major and Sally's brother is a math major.
		\\ \ans Sally is not a CS major or Sally's brother is not a math major.
		\item Either the professor is late or my watch is fast.
		\\ \ans The professor is not late and my watch is not fast.
		\\ \ans OR \ \ The professor is on time (or early) and my watch is accurate (or slow).
	\end{enumerate}
	
	\item For each of the following statements, give the contrapositive, converse, and
	inverse statements (label them) in normal English, using the syntax ``If \ldots, then \ldots''  
	You may change verb tenses to improve
	the grammar.
	\begin{enumerate}
		\item ``If you conquer yourself, then you conquer the world.''\footnote{From \emph{Aleph}, by Paulo Coelho.}			
		\item I will be able to retire if I save enough money.
		\item You can go to the party only if you get good grades.

		
	\end{enumerate}
	
	\textbf{Solution:} 
        \begin{enumerate}
                \item Original statement: If you conquer yourself, then you conquer the world. \\
                Contrapositive: If you don't conquer the world, then you didn't conquer yourself. \\
                Converse: If you conquer the world, then you conquer yourself.\\
                Inverse: If you don't conquer yourself, you won't conquer the world. 

                
                \item Original statement: I will be able to retire if I save enough money.. \\
                Equivalent original statement, rephrased: If I save enough money, then I will be able to retire. \\
                Contrapositive: If I'm not able to retire, then I didn't save enough money. \\
                Converse: If I'm able to retire, then I saved enough money. \\
                Inverse: If I don't save enough money, then I won't be able to retire.
                
                \item Original statement: You can go to the party only if you get good grades..  \\
                Equivalent original statement, rephrased: If you don't get good grades, then you can't go to the party. \\
                Contrapositive: If you go to the party, then you must have gotten good grades.\\
                Converse: If you don't go to the party, then you didn't get good grades.\\
                Inverse: If you get good grades, then you can go to the party.
                
                \emph{Note:} The statement ``$p$ only if $q$'' means ``if not $q$ then not $p$,'' as well as ``if $p$ then $q$'' (see Epp page 44);
                clearly $\nott q \to \nott p \equiv p \to q$.  Therefore, it's OK if you switched the original statement and the contrapositive above (because
                those are logically equivalent to each other) as well as the converse and inverse (the converse and inverse are also always logically equivalent
                to each other).

        \end{enumerate}

	
	\item Let $x$, $y$ and $z$ be statements.  Construct a complete truth table\\ for the 
	statement $(x \to \nott y) \land (\nott x \lor z)$.
	
	\begin{tabular}{|l|l|l|l|l|l|l|l|} \hline
        $x$ & $y$ & $z$ & $\nott y$ & $x \to \nott y$ & $\nott x$ & $\nott x \lor z$ & $(x \to \nott y) \land (\nott x \lor z)$  \\ \hline
         T & T & T & F & F & F & T & F \\ \hline
         T & T & F & F & F & F & F & F \\ \hline
         T & F & T & T & T & F & T & T \\ \hline
         T & F & F & T & T & F & F & F \\ \hline
         F & T & T & F & T & T & T & T \\ \hline
         F & T & F & F & T & T & T & T \\ \hline
         F & F & T & T & T & T & T & T \\ \hline
         F & F & F & T & T & T & T & T \\ \hline
	\end{tabular}

	
	\item Are the statements $\nott (p \lor q)$ and $\nott p \lor \nott q$ logically equivalent?  Use a complete truth table to justify
	your answer, and explain (in English) why the truth table supports your answer.
	
	\ans No, the statements are not equivalent.
	
	\begin{tabular}{|l|l|l|l|l|l|l|} \hline
        $p$ & $q$ & $p \lor q$ & $\nott (p \lor q)$ & $\nott p$ & $\nott q$ & $\nott p \lor \nott q$   \\ \hline
         T & T & T & F & F & F & F \\ \hline
         T & F & T & F & F & T & T \\ \hline
         F & T & T & F & T & F & T \\ \hline
         F & F & F & T & T & T & T \\ \hline
	\end{tabular}
	
	The truth table supports the fact that the two statements are not equivalent because the columns for 
	$\nott (p \lor q)$ and $\nott p \lor \nott q$
	are different (they disagree on true/false for the two middle rows).

\end{enumerate}


\end{document}
