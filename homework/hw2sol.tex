\documentclass[12pt, letterpaper]{report}
\usepackage{amsmath,fullpage,graphicx}

\setlength{\parindent}{0em}
\newcommand{\nott}{{\sim}}
\newcommand{\Z}{\mathbb{Z}}
\newcommand{\powerset}[1]{\mathcal P \left({#1}\right)}
\newcommand{\proofnote}[1]{[\textit{Note: #1}]}
\newcommand{\sol}{\textbf{Solution: }}


\begin{document}

{\textbf{Discrete Structures, Fall 2017, Homework 2 Solutions}}

\medbreak

You must write the solutions to these problems legibly on your own paper, with
the problems in sequential order, and with all sheets stapled together.

\medbreak

\begin{enumerate}

	\item Construct a complete truth table to help you determine if the following 
	argument is valid or not. State whether it is valid or not, indicate the critical rows in the truth table, 
	and explain why those rows support your answer.
	
	\begin{tabular}{ll}
		Premise: & $\nott q \lor (p \land \nott r)$ \\
		Premise: & $\nott r \to p$ \\
		Conclusion: & $q \to r$
	\end{tabular}
	
	     \sol      
	       
        \begin{tabular}{|l|l|l|l|l|l|l|l|l|l|} \hline
        &&&&&&Premise&Premise&Conclusion&\\
        $p$ & $q$ & $r$ & $\nott q$ & $\nott r$ & $p \land \nott r$ & $\nott q \lor (p \land \nott r)$ & $\nott r \to p$ & $q \to r$ & \\ \hline
         T & T & T & F & F & F & F & T & T &\\ \hline
         T & T & F & F & T & T & T & T & F &$\gets$ critical row\\ \hline
         T & F & T & T & F & F & T & T & T &$\gets$ critical row\\ \hline
         T & F & F & T & T & T & T & T & T &$\gets$ critical row\\ \hline
         F & T & T & F & F & F & F & T & T &\\ \hline
         F & T & F & F & T & F & F & F & F &\\ \hline
         F & F & T & T & F & F & T & T & T &$\gets$ critical row\\ \hline
         F & F & F & T & T & F & T & F & T &\\ \hline

        \end{tabular}
        
         The critical rows are the rows of the truth table where both premises
        are true.  Because the conclusion is not true in \emph{all} of the critical rows,  the argument
        is invalid.  (The conclusion is false in the first labeled critical row.)

	
	\item For each of the following, if a \emph{single}, valid rule of inference can lead from the given premises to the given conclusion, state what
	rule of inference would be used. If no valid rule could be used, write ``no rule.''

	\begin{enumerate}
		\item 
		Premise:  $(m \to p) \lor (n \to p)$ \\
		Premise:  $n$ \\
		Conclusion: $p$ \\
		
		\sol No rule. (Not a valid argument anyway.)                  \bigskip

		\item 
		Premise:  $(a \land b) \to (z \lor y)$ \\
		Premise:  $\nott (z \lor y)$ \\
		Conclusion: $\nott (a \land b)$  \\
		
		\sol Modus tollens.

		\item 
		Premise:  $r \land (q \lor p)$ \\
		Premise:  $\nott q$ \\
		Conclusion: $p$  \\
		
		\sol No rule.  (It's a valid argument, but you would need to use conjunctive simplification first, then disjunctive syllogism.)\bigskip
		
		\item 
		Premise:  $k \land m$ \\
		Conclusion: $(k \land m) \lor (k \to (n \to m))$ \\
		
		\sol Disjunctive addition.\bigskip
		
		\item 
		Premise:  $r \land (q \lor p)$ \\
		Premise:  $\nott q$ \\
		Conclusion: $r \land p$ \\
		
		\sol No rule.  (It's a valid argument, but just like part (c), you'd need to use conjunctive simplification, then disjunctive syllogism, then
		conjunctive addition.) \bigskip
		
		\item 
		Premise: $g \to h$\\
		Premise: $(e \land k) \lor g$ \\
		Premise: $(e \land k) \to h$ \\
		Conclusion: $h$
		
		\sol Dilemma (or proof by division into cases).
	\end{enumerate}


	\item Complete the following proofs using the framework discussed in class.  Each line
	of your proof must be justified with a rule of inference or logical equivalence and appropriate line numbers.
	
	\begin{enumerate}
		\item
			\begin{tabular}[t]{ll}
			P1 & $\nott p$ \\ 
			P2 & $(q \land \nott p) \to m$ \\
			P3 & $\nott r \lor q$ \\ 
			P4 & $r$ \\ \hline
			Prove: & $m$
		\end{tabular}
		
		\sol
		
		\begin{tabular}{|r|l|c|c|} \hline
                Line & Statement & Rule & Lines Used \\ \hline
                1 & $q$ & Disjunctive syllogism & P3, P4 \\ \hline
                2 & $q \land \nott p$ & Conjunctive addition & 1, P1 \\ \hline
                3 & $m$ & Modus ponens & P2, 2 \\ \hline
                \end{tabular}
		
		\item
			\begin{tabular}[t]{ll}
			P1 & $\nott k \land g$ \\ 
			P2 & $m \to \nott g$ \\
			P3 & $(\nott f \to k) \lor m$ \\ \hline
			Prove: & $f$
		\end{tabular}
		
		\sol
		
\begin{tabular}{|r|l|c|c|} \hline
                Line & Statement & Rule & Lines Used \\ \hline
                1 & $\nott k$ & Conjunctive simplification & P1 \\ \hline
                2 & $g$ & Conjunctive simplification & P1 \\ \hline
                3 & $\nott m$ & Modus tollens & 2, P2 \\ \hline
                4 & $\nott f \to k$ & Disjunctive syllogism & 3, P3 \\ \hline
                5 & $f$ & Modus tollens & 1, 4 \\ \hline
        \end{tabular}
		
	\end{enumerate}

\end{enumerate}


\end{document}
