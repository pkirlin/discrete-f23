\documentclass[11pt, letterpaper]{report}
\usepackage{amsmath,fullpage,graphicx,amssymb}

\setlength{\parindent}{0em}

\newcommand{\nott}{{\sim}}

\newcommand{\Z}{\mathbb{Z}}
\newcommand{\proofnote}[1]{[\textit{Note: #1}]}
\newcommand{\powerset}[1]{\mathcal P\left({#1}\right)}


\begin{document}

\textbf{Discrete Structures, Fall 2017, Homework 9}

\bigskip

You must write the solutions to these problems legibly on your own paper, with
the problems in sequential order, and with all sheets stapled together.

\begin{enumerate}

	\item Prove the following statement using an \textit{element proof}:
	
	For any four sets $A$, $B$, $C$, and $D$, if $C \subseteq (D \cup A)$ and $B \subseteq D^c$, then
	$C \cap B \subseteq A$.
	
	
	        
        \textbf{Proof:}
        
        Let $A$, $B$, $C$, and $D$ be arbitrary sets.
        
        Assume $C \subseteq (D \cup A)$ and $B \subseteq D^c$.
        
        \proofnote{We are trying to prove that $C \cap B \subseteq A$, so let's assume $x$ is an arbitrary element in the left side of the subset
        ($C \cap B$) and show $x$ must be in the right side ($A$).}
        
        Let $x$ be an arbitrary element in $U$.  Assume $x \in C \cap B$.  
        
        \proofnote{The line above could also be written as ``Let $x$ be an arbitrary element in $C \cap B$.''}
        
        $x \in C \cap B$ \qquad (from above)
        
        $x \in C \land x \in B$ \qquad (def of intersection)
        
        $x \in C$ \qquad (conjunctive simplification)
        
        $x \in B$ \qquad (conjunctive simplification)
        
        $x \in D \cup A$ \qquad (because $C \subseteq (D \cup A)$ and $x \in C$)
        
        $x \in D \lor x \in A$ \qquad (def of union)
        
        $x \in D^c$ \qquad (because $B \subseteq D^c$ and $x \in B$)
        
        $x \not\in D$ \qquad (def of set complement)
        
        $x \in A$ \qquad (disjunctive syllogism: note $x \in D \lor x \in A$ but $x \not\in D$)
        
        Because we assumed $x$ was an arbitrary element in $C \cap B$ and we showed $x \in A$, we can conclude (by the definition of subset)
        that $C \cap B \subseteq A$.
                

	
	
	 \item Prove the following statement using an \textit{element proof}:
        
        For any three sets $A$, $B$, and $C$, if $A \cap C \subseteq B$, then
        $(A-B) \cap (C-B) = \emptyset$.
        
        
        \textbf{Proof:}
                
        Let $A$, $B$, and $C$ be arbitrary sets, and assume that $A \cap C \subseteq B$.
        
        Assume by way of contradiction that $\exists x \in U \ x \in (A-B) \cap (C-B)$.
        
        $(x \in A-B) \land (x \in C-B)$ by definition of intersection.
        
        $(x \in A \land x \not\in B) \land (x \in C \land x \not\in B)$ by definition of set difference.
        
        $x \in A \land x \not\in B \land x \in C$ by associativity, and removing one of the $\not\in B$'s (idempotent rule).
        
        $x \in A \land x \in C$ by conjunctive simplification.% (and commutativity).
        
        $x \not\in B$ by conjunctive simplification.
        
        $x \in (A \cap C)$ by definition of intersection.
        
        $x \in B$ because $x \in A \cap C$ and $A\cap C \subseteq B$.
        
        Now we have a contradiction between $x \in B$ and $x \not\in B$, so therefore our
        assumption about there being an element in $(C-A) \cap (B-A)$ must have been wrong.
        
        Therefore, we know $\forall x \in U \ x \not\in (A-B) \cap (C-B)$.
        
        Therefore, by the definition of the empty set, $(A-B) \cap (C-B) = \emptyset$.
        
                

        
        \item Prove the following statement using an \textit{element proof}:
        
        For any three sets $A$, $B$, and $C$, if $A \cup C \subseteq B$, then
        $C \times A \subseteq B \times B$.
        
        
        
        \textbf{Proof:}
        
        Let $A, B, $ and $C$ be arbitrary sets, and assume that $A \cup C \subseteq B$.
        
        \proofnote{We want to show $C \times A \subseteq B \times B$, so we'll do this with the standard method to prove a set is a subset of another set.  We'll pick an arbitrary
        element in $C \times A$ and show that element must also be in $B \times B$.}
        
        Let $(x,y)$ be an arbitrary element in $C \times A$.
        
        $x\in C \land y \in A$ by definition of Cartesian product.
        
        $x \in C$ by conjunctive simplification.
        
        $y \in A$ by conjunctive simplification.
        
        \proofnote{At this point, our goal is to show $(x,y) \in B \times B$, which means showing that
        $x \in B$ and $y \in B$.  We know $A \cup C \subseteq B$, which means if we can show $x$ is an element of $A \cup C$, we will be able to deduce that $x$ is a member of $B$ (by definition of subset).  The same argument goes for $y$. 
        \vspace{.2in}\\
        For $x$ to be in the union of $A$ and $C$, $x$ must be in either $A$ or $C$.  We already know $x \in A$,
        so to build the statement ``$x \in A \lor x \in C$'' we will use the rule of disjunctive addition.}
        
        $x \in A \lor x \in C$ by disjunctive addition \emph{[note: $x \in A$ from above, so $x \in A \lor \text{whatever}$ is true no matter what the ``whatever'' part is]}
        
        $x \in A \cup C$ by definition of union
        
        $x \in B$ because $x \in A \cup C$ and $A\cup C \subseteq B$.
        
        $y \in A \lor y\in C$ by disjunctive addition \emph{[note: $y \in A$ from above, so $\text{whatever} \lor y \in A$ is true no matter what the ``whatever'' part is]}
        
        $y \in A \cup C$ by definition of union
        
        $y \in B$ because $y \in A \cup C$ and $A\cup C \subseteq B$.
        
        $x \in B \land y \in B$ by conjunctive addition.
        
        $(x,y) \in B \times B$ by definition of Cartesian product.
        
        %$(x,y) \in A \times B \to (x,y) \in B \times C$ by closing conditional world.
        
        %$\forall (x,y) \in U \ [(x,y) \in A \times B \to (x,y) \in B \times C]$ by generalizing from
        %the generic particular (universal generalization).
        
        %$A \times B \subseteq B \times C$ by definition of subset.
        
        Because we have shown that if we choose an arbitrary element of $C \times A$,
        then that element must also be in $B \times B$, then we can conclude that
        $C \times A \subseteq B \times B$.
        

  \item Prove the following statement using an \textit{algebraic proof}:
        
        For any two sets $A$ and $B$, $(B \cup (B-A^c))^c = B^c$.
      
           
        
        \proofnote{We can do an algebraic proof here, because there are no initial 
        assumptions about $A$ or $B$ (like one being a subset
        of the other or something similar).  If there were any assumptions, we would need to do element proofs for subset in both directions to show equality.}
        
        
        \textbf{Proof:}
        \begin{align*}
                (B \cup (B-A^c))^c &= (B \cup (B \cap (A^{c})^c))^c & \text{def of set difference} \\
                &= (B \cup (B \cap A))^c & \text{double complement law*} \\
                &= (B)^c = B^c & \text{absorption law} \\
        \end{align*}
        Note how we used absorption to change $B \cup (B \cap A)$ into just $B$ (and the
        complement was kept around the parentheses).
        
        * OK to combine double complement with def of set difference from line above.
        
        Check the cheatsheet!  Absorption is one of those rules that doesn't come up very often.
        
        In this proof, you could also do DeMorgan's before the absorption.  In that case,
        you would have ended up with a line that looks like $B^c \cap (B \cap A)^c$, which
        you could do another DeMorgan's on to get $B^c \cap (B^c \cup A^c)$.  And then, on
        that last line, you can do absorption to get $B^c$.
        
        
        
           \item    
\begin{enumerate}
                \item Define $S = \{x,y,z\}$.  What is $\powerset{S}$?
                
                \textbf{Solution}: $\powerset{S} = \{\emptyset, \{x\}, \{y\}, \{z\}, \{x,y\},\{x,z\},\{y,z\}, \{x,y,z\}\}$
                \item What is $\powerset{\emptyset}$?
                
                \textbf{Solution}: $\powerset{\emptyset}= \{\emptyset\}$ \\
                Note that this set is \emph{not} the empty set; it has one element in it!
                
                We can do a quick sanity check here.  We know that for any set $S$, the size of $\powerset{S}$
                should be equal to 2 raised to the power of the size of $S$.  So because the size of $\emptyset$
                is zero, the size of  $\powerset{\emptyset}$ should be $2^0=1$.

                \item What is $\powerset{\powerset{\emptyset}}$?
                
                \textbf{Solution}: $\powerset{\emptyset}= \{\emptyset , \{\emptyset\}\}$ 
                
                This is the powerset of the set from part (b), so it should have $2^1=2$ elements, which it does:
                the two elements are the empty set, and the set containing the empty set.
        \end{enumerate}

\end{enumerate}

\end{document}
