\documentclass[12pt, letterpaper]{report}
\usepackage{amsmath,fullpage,graphicx,amssymb}

\setlength{\parindent}{0em}

\newcommand{\nott}{{\sim}}

\newcommand{\Z}{\mathbb{Z}}

\begin{document}

{\textbf{Discrete Structures, Fall 2023, Homework 8}}

\bigskip

You must write the solutions to these problems legibly on your own paper, with
the problems in sequential order, and with all sheets stapled together.

\bigskip

Prove the following statements by strong induction.  Make sure to follow the form from class:
explicitly define $P(n)$, label the base case(s), inductive case/step, where you define the inductive hypothesis, where you define what you want to 
prove,
and where you use the inductive hypothesis.  
\begin{enumerate}

\item Suppose we define a sequence as follows: 
$$
a_1 = 3; \ a_2 = 5; \text{ and for all integers }i \geq 3, \ a_i = 4a_{i-1} -3a_{i-2}. 
$$
\textbf{Prove} $\forall n \in \Z^{\geq 1} \ a_n = 3^{n-1}+2.$

\item Suppose we define a sequence as follows: 
$$
b_0 = 3; \ b_1 = 1; \ b_2 = 3; \text{ and for all integers }i \geq 3, \ b_i = b_{i-3} + b_{i-2}  +b_{i-1}. 
$$
\textbf{Prove} that every term in the sequence is odd.


\item Suppose we define a sequence as follows: 
$$
f_1 = 1; \ f_2=1; \text{ and for all integers }i \geq 3, \ f_i = f_{i-1}  + f_{i-2}.
$$
This is one definition of the Fibonacci sequence, which you will remember from class, and probably COMP 142.  

\textbf{Prove} $\forall n \in \Z^{\geq 6} \ f_n \geq (\sqrt{2})^n$.

Hint: Remember that $\sqrt{2} = 2^{1/2}$.  So therefore $(\sqrt{2})^n = (2^{1/2})^n = (2^{n/2})$. This may help you if you prefer working with a base
of $2$ rather than $\sqrt{2}$.  But you can do the proof with either base.

Hint 2: Remember $b^{x+y} = b^x \cdot b^y$, and $b^{x-y} = b^x/b^y$, and $b^{xy} = (b^x)^y = (b^y)^x$.

Side note: In class, we proved $f_n < 2^n$.  This proof is similar, except the inequality is reversed, which illustrates that
the Fibonacci sequence grows exponentially fast, but the base of the exponential is somewhere between $\sqrt{2}$ and $2$.
(The actual base is the \emph{golden ratio}, which is $\frac{1+\sqrt{5}}{2} \approx 1.618$.)

\item \phantom{} [Optional, 2 points extra credit] \\ In problem 3, why does the proof specify ``6'' in $\forall n \in \Z^{\geq 6}$?





\end{enumerate}

\end{document}
