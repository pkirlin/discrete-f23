\documentclass[11pt, letterpaper]{report}
\usepackage{amsmath,fullpage,graphicx,amssymb}

\setlength{\parindent}{0em}

\newcommand{\nott}{{\sim}}

\newcommand{\Z}{\mathbb{Z}}

\begin{document}

{\textbf{Discrete Structures, Fall 2016, Homework 7 Solutions}}

\medbreak

\textit{You must write the solutions to these problems legibly on your own paper, with
the problems in sequential order, and with all sheets stapled together.}

\bigskip

Prove each of the following statements using ``regular'' or ``weak'' induction.

\begin{enumerate}

       
\item $\forall n \in \Z^{\geq 0} \ \displaystyle \sum_{i=0}^n (3i^2-i) = n^2(n+1)$      
       
       
\textbf{Solution:}

Define $P(n)$ as $\displaystyle \sum_{i=0}^n (3i^2-i) = n^2(n+1)$.

\textbf{Base Case:}

Prove $P(0)$, which means prove $\displaystyle \sum_{i=0}^0 (3i^2-i) = 0^2(0+1)$.

LHS = $\displaystyle \sum_{i=0}^0 (3i^2-i) = 3(0)^2-0=0$.

RHS = $\displaystyle 0^2(0+1)=0$.

LHS = RHS \qquad \textit{(base case proved)}

\textbf{Inductive Case:}

Suppose $k$ is an arbitrary integer in $ \Z^{\geq 0}$.

\medskip

Inductive Hypothesis: Assume that $P(k)$ is true; \\in other words, assume
$\displaystyle \sum_{i=0}^k (3i^2-i) = k^2(k+1)$. 

Prove that $P(k+1)$ is true; \\ \medskip in other words, prove
$\displaystyle \sum_{i=0}^{k+1} (3i^2-i) = (k+1)^2((k+1)+1)$

\textit{To prove this equation is true, we will start with the left side and transform it 
into the right side.}
\begin{align*}
\text{LHS}=\sum_{i=0}^{k+1} (3i^2-i) & = \sum_{i=0}^{k} (3i^2-i) + 3(k+1)^2-(k+1) & \text{manipulation of a summation}\\
& = k^2(k+1) + 3(k+1)^2-(k+1) & \text{by IH}\\
& = (k+1)[k^2 + 3(k+1)-1] & \text{algebra: factor out $k+1$}\\
& = (k+1)[k^2 + 3k+3-1] & \text{algebra}\\
& = (k+1)[k^2 + 3k+2] & \text{algebra}\\
& = (k+1)[(k+1)(k+2)] & \text{algebra}\\
& = (k+1)^2[k+2] & \text{algebra}\\
& = (k+1)^2[(k+1)+1]=\text{RHS}& \text{algebra}
\end{align*}
\textit{(inductive case proved)}


\item Prove $\displaystyle \forall n \in \Z^+ \ \prod_{i=1}^n i(i+1) = (n+1)(n!)^2$

Hint: Recall that $n!=n(n-1)(n-2)\cdots2\cdot 1$, with $0!$ defined to be 1.  However, an alternate
formula involving recursion is the following:
$$n! = \begin{cases} 1 &\text{ if } n= 0 \\
n \cdot (n-1)! &\text{ otherwise}\end{cases}$$
This recursive definition will be useful during the inductive step.     


\textbf{Solution:}

Define $P(n)$ as $\displaystyle  \prod_{i=1}^n i(i+1) = (n+1)(n!)^2$.

\textbf{Base Case:}

Prove $P(1)$, which means prove $\displaystyle  \prod_{i=1}^1 i(i+1) = (1+1)(1!)^2$.

LHS = $\displaystyle \prod_{i=1}^1 i(i+1) = 1(1+1)=2$.

RHS = $\displaystyle (1+1)(1!)^2=2(1)^2=2$.

LHS = RHS \qquad \textit{(base case proved)}

\textbf{Inductive Case:}

Suppose $k$ is an arbitrary integer in $ \Z^{+}$.

\medskip

Inductive Hypothesis: Assume that $P(k)$ is true; \\in other words, assume
$\displaystyle \prod_{i=1}^k i(i+1) = (k+1)(k!)^2$. 

Prove that $P(k+1)$ is true; \\ \medskip in other words, prove
$\displaystyle \prod_{i=1}^{k+1} i(i+1) = ((k+1)+1)((k+1)!)^2$.

\textit{To prove this equation is true, we will start with the left side and transform it 
into the right side.}

\begin{align*}
\text{LHS}=\prod_{i=1}^{k+1} i(i+1) & = \prod_{i=1}^{k} i(i+1) \cdot (k+1)((k+1)+1) & \text{manipulation of a product}\\
& = (k+1)(k!)^2 \cdot (k+1)(k+2) & \text{by IH}\\
& = (k+1)^2 \cdot (k!)^2 \cdot (k+2) & \text{algebra}\\
& = [(k+1) \cdot (k!)]^2 \cdot (k+2) & \text{algebra}\\
\intertext{Now note that by the recursive definition of factorial, $(k+1)! = (k+1)\cdot k!$}
& = [(k+1)!]^2 \cdot (k+2) & \text{algebra}\\
& = [(k+1)!]^2 \cdot ((k+1)+1)=\text{RHS} & \text{algebra}
\end{align*}
\textit{(inductive case proved)}




\item $\forall n \in \Z^{\geq 0} \ n(n+1)$ is even.

Note: This is the same problem as question 1 on the last homework (prove that the product of any two consecutive integers is even). 
In homework 7, you did this with the QRT.  On this homework, you should use induction (do not use the QRT here).


\textbf{Solution:}

Define $P(n)$ as ``$n(n+1)$ is even''.

\textbf{Base Case:}

Prove $P(0)$, which means prove $0(0+1)$ is even.

We need to prove $0\cdot 1 = 0$ is even.  Zero is clearly even.
 \textit{(base case proved)}

\textbf{Inductive Case:}

Suppose $k$ is an arbitrary integer in $ \Z^{\geq 0}$.

\medskip

Inductive Hypothesis: Assume that $P(k)$ is true; \\in other words, assume
$k(k+1)$ is even.

Prove that $P(k+1)$ is true; \\ \medskip in other words, prove
$(k+1)((k+1)+1)$ is even.

\textit{To prove this is true, we will prove $(k+1)(k+2)$ is even by using the definition of even --- we will try
to get an equation that looks like $(k+1)(k+2) = 2*(some\ integer)$.}

\begin{align*}
(k+1)((k+1)+1) &=(k+1)(k+2) & \text{(by algebra)}\\
&= k^2+k+2k+2 & \text{(by algebra)}\\
\intertext{By the IH, we know $k(k+1)$ is even.  Therefore, by the definition of even, there exists some integer $p$
such that $k(k+1)=2p$. By algebra, $k(k+1)=k^2+k=2p$.}
(k+1)((k+1)+1) &= k^2+k+2k+2 & \text{(from above)}\\
 &= 2p+2k+2 & \text{(substitution)}\\
  &= 2(p+k+1) & \text{(algebra)}
\end{align*}
$p+k+1$ is an integer by closure of the integers under addition.  Therefore, $(k+1)((k+1)+1)$  is even
by the definition of even.

\textit{(inductive case proved)}


\item $\forall n \in \Z^{\geq 0} \ \ 5 \mid 7^n - 2^n$.   

\textbf{Solution:}

Define $P(n)$ as $5 \mid 7^n - 2^n$.

\textbf{Base Case:}

Prove $P(0)$, which means prove $5 \mid 7^0 - 2^0$

$7^0-2^0 = 1-1=0$.  $5 \mid 0$.  \ \ \checkmark \ \
 \textit{(base case proved)}

\textbf{Inductive Case:}

Suppose $k$ is an arbitrary integer in $ \Z^{\geq 0}$.

\medskip

Inductive Hypothesis: Assume that $P(k)$ is true; \\in other words, assume
$5 \mid 7^k - 2^k$.

Prove that $P(k+1)$ is true; \\ \medskip in other words, prove
$5 \mid 7^{k+1} - 2^{k+1}$.

\textit{[To prove this is true, we will prove $5 \mid 7^{k+1} - 2^{k+1}$  by using the definition of divides --- we will try
to get an equation that looks like $7^{k+1} - 2^{k+1}$ = 5*(some\ integer).]}

\begin{align*}
7^{k+1} - 2^{k+1} &=  7\cdot 7^{k} - 2\cdot 2^{k}  & \text{(by algebra)}\\
\intertext{By the IH, we know $5 \mid 7^k - 2^k$.  Therefore, by the definition of divides, there exists some integer $p$
such that $7^k - 2^k=5p$. By algebra, $7^k=5p+2^k$.}
7^{k+1} - 2^{k+1} &=  7\cdot 7^{k} - 2\cdot 2^{k}  & \text{(from above)}\\
&= 7\cdot (5p+2^k) - 2\cdot 2^{k} & \text{(substitution)}\\
&= 35p + 7\cdot 2^k - 2\cdot 2^{k} & \text{(algebra)}\\
&= 35p + 2^k(7- 2) & \text{(algebra)}\\
&= 35p + 2^k(5) & \text{(algebra)}\\
&= 5(7p + 2^k) & \text{(algebra)}
\end{align*}
$7p + 2^k$ is an integer by closure of the integers under addition and multiplication.  Therefore, $5\mid 7^{k+1} - 2^{k+1}$ 
by the definition of divides.

\textit{(inductive case proved)}


\end{enumerate}

\end{document}