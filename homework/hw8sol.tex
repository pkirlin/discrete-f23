\documentclass[11pt, letterpaper]{report}
\usepackage{amsmath,fullpage,graphicx,amssymb}

\setlength{\parindent}{0em}

\newcommand{\nott}{{\sim}}

\newcommand{\Z}{\mathbb{Z}}

\begin{document}

{\textbf{Discrete Structures, Fall 2017, Homework 8 Solutions}}

\bigskip

You must write the solutions to these problems legibly on your own paper, with
the problems in sequential order, and with all sheets stapled together.

\bigskip

Prove the following statements by strong induction.  Make sure to follow the form from class:
explicitly define $P(n)$, label the basis step(s), inductive step, where you define the inductive hypothesis, where you define what you want to prove
and where you use the inductive hypothesis.  
\begin{enumerate}

\item Suppose we define a sequence as follows: 
$$
a_1 = 3; \ a_2 = 5; \text{ and for all integers }i \geq 3, \ a_i = 4a_{i-1} -3a_{i-2}. 
$$
Prove $\forall n \in \Z^{\geq 1} \ a_n = 3^{n-1}+2.$

\textbf{Solution:}

Define $P(n)$ as ``$a_n = 3^{n-1}+2.$''  We are trying to prove $\forall n \in \Z^{\geq 1} \ P(n)$.

Base Cases:

Prove $P(1)$ and $P(2)$.  That is, prove that $a_1 = 3^{1-1}+2$ and $a_2 = 3^{2-1}+2$.

LHS = $a_1 = 3$.  RHS = $3^{1-1}+2 = 3^0+2 = 1+2=3$.  LHS = RHS. \ \ \checkmark 

LHS = $a_2 = 5$.  RHS = $3^{2-1}+2 = 3^1+2 = 3+2=5$.  LHS = RHS. \ \ \checkmark 

\textit{(base cases proved)}

Inductive Case:

Suppose $k$ is an arbitrary integer $\geq 2$.

Inductive hypothesis: Assume for all integers $i$, $1 \leq i \leq k$, that $P(i)$ is true.\\
In other words, assume $P(1) \land P(2) \land P(3) \land \cdots \land P(k)$.\\
In other words, assume $a_1 = 3^{1-1}+2$ and $a_2 = 3^{2-1}+2$ and $a_3 = 3^{3-1}+2$ and \ldots\ and $a_k = 3^{k-1}+2$.

Inductive step: Prove $P(k+1)$ is true.  In other words, prove that $a_{k+1} = 3^{k+1-1}+2$.

\begin{align*}
a_{k+1} &= 4a_{k} - 3a_{k-1} &\text{definition of the sequence}\\
\intertext{By the inductive hypothesis, $P(k-1)$ and $P(k)$ are both true.  Therefore, $a_{k-1} = 3^{k-1-1}+2$ and $a_k = 3^{k-1}+2$.  Substitute these equations into our equation above:}
a_{k+1} &= 4a_{k} - 3a_{k-1} &\text{from above}\\
&= 4(3^{k-1}+2) - 3(3^{k-1-1}+2) &\text{substitution}\\
&= 4\cdot 3^{k-1}+8 - 3\cdot 3^{k-2}-6 &\text{algebra}\\
&= 4\cdot 3^{k-1}- 3\cdot 3^{k-2}+2 &\text{algebra}\\
&= 4\cdot 3^{k-1}-  3^{k-1}+2 &\text{algebra}\\
&= 3\cdot 3^{k-1} +2 = 3^{k}+2 = 3^{k+1-1}+2&\text{algebra}
\end{align*}
\textit{[Inductive case proved.]}

\item Suppose we define a sequence as follows: 
$$
b_0 = 3; \ b_1 = 1; \ b_2 = 3; \text{ and for all integers }i \geq 3, \ b_i = b_{i-3} + b_{i-2}  +b_{i-1}. 
$$
Prove that every term in the sequence is odd.

\textbf{Solution:}

Define $P(n)$ as ``$b_n$ is odd.''  We are trying to prove $\forall n \in \Z^{\geq 0} \ P(n)$.

Base Cases:

Prove $P(0)$, $P(1)$, and $P(2)$.  That is, prove that $b_0$, $b_1$, and $b_2$ are all odd.

$b_0 = 3$.  3 is odd. \ \ \checkmark 

$b_1 = 1$.  1 is odd. \ \ \checkmark 

$b_2 = 3$.  3 is odd. \ \ \checkmark 

\textit{(base cases proved)}

Inductive Case:

Suppose $k$ is an arbitrary integer $\geq 2$.

Inductive hypothesis: Assume for all integers $i$, $0 \leq i \leq k$, that $P(i)$ is true.\\
In other words, assume $P(0) \land P(1) \land P(2) \land \cdots \land P(k)$.\\
In other words, assume $b_1$ is odd and $b_2$ is odd and $b_3$ is odd and \ldots and $b_k$ is odd.

Inductive step: Prove $P(k+1)$ is true.  In other words, prove that $b_{k+1}$ is odd.

\textit{[We will prove $b_{k+1}$ is odd using the def'n of odd.  That is, we will try to get an equation that
looks like $b_{k+1}=2*(some\  integer)+1]$}.


\begin{align*}
b_{k+1} &= b_{k-2} + b_{k-1}  +b_{k} &\text{definition of the sequence}\\
\intertext{By the inductive hypothesis, $P(k-2)$, $P(k-1)$ and $P(k)$ are all true.  Therefore, $b_{k-2}$, $b_{k-1}$
and $b_k$ are all odd.}
\intertext{Therefore, by the definition of odd, there exist integers $p, q, r$ such that $b_{k-2}=2p+1$, $b_{k-1}=2q+1$,
and $b_{k}=2r+1$.  Let's substitute these in:}
b_{k+1} &= b_{k-2} + b_{k-1}  +b_{k} &\text{from above}\\
&= (2p+1) + (2q+1) + (2r+1) &\text{definition of the sequence}\\
&= 2p+2q+2r+2+1 & \text{algebra}\\
&= 2(p + q + r + 1) + 1 & \text{algebra}
\end{align*}
$p + q + r + 1$ is an integer by closure of the integers under addition.  Therefore, $b_{k+1}$ is odd by the definition of odd.
\textit{[Inductive case done.]}

\item Suppose we define a sequence as follows: 
$$
c_0 = 2; \ c_1=7; \text{ and for all integers }i \geq 2, \ c_i = 3c_{i-1} -2c_{i-2}. 
$$
Prove $\forall n \in \Z^{\geq 0} \ 5 \mid (c_n - 2)$.

\textbf{Solution:}

Define $P(n)$ as ``$5\mid (c_n-2)$.''  We are trying to prove $\forall n \in \Z^{\geq 0} \ P(n)$.

Base Cases:

Prove $P(0)$ and $P(1)$.  That is, prove that $5 \mid (c_0-2)$ and $5 \mid (c_1-2)$.

$c_0 -2= 2-2=0$.  $5\mid 0$.  \ \ \checkmark 

$c_1 -2= 7-2=5$.  $5\mid 5$.  \ \ \checkmark 

\textit{(base cases proved)}

Inductive Case:

Suppose $k$ is an arbitrary integer $\geq 1$.

Inductive hypothesis: Assume for all integers $i$, $0 \leq i \leq k$, that $P(i)$ is true.\\
In other words, assume $P(0) \land P(1) \land P(2) \land \cdots \land P(k)$.\\
In other words, assume $5 \mid (c_0-2)$ and $5 \mid (c_1-2)$ and $5 \mid (c_2-2)$ and \ldots and $5 \mid (c_k-2)$.



Inductive step: Prove $P(k+1)$ is true.  In other words, prove that $5\mid (c_{k+1}-2)$.

\textit{[We will prove $5\mid (c_{k+1}-2)$ using the def'n of divides.  That is, we will try to get an equation that
looks like $c_{k+1}-2=5*(some\  integer)]$}.

\begin{align*}
c_{k+1} &= 3c_{k} - 2c_{k-1} &\text{definition of the sequence}\\
\intertext{By the inductive hypothesis, $P(k-1)$ and $P(k)$ are both true.  Therefore, $5\mid (c_{k-1} - 2)$ and $5\mid (c_k-2)$.}
\intertext{Therefore, by the definition of divides, there exist integers $q,r \in \Z$ such that 
$(c_{k-1}-2)=5q$ and $(c_k-2)=5r$.}
\intertext{Let's add 2 to both sides of both equations.  We get $c_{k-1}=5q+2$ and $c_k=5r+2$.   Substitute these into
our earlier equation:}
c_{k+1} &= 3c_{k} - 2c_{k-1} = 3(5q+2)-2(5r+2)&\text{substitution}\\
&= 15q+6-10r-4=5(3q-2r)+2  & \text{all by algebra}
\intertext{Now subtract 2 from both sides:}
c_{k+1} -2 &= 5(3q-2r)+2 & \text{algebra}
\end{align*}
Because the integers are closed under multiplication and addition, we know $3q-2r \in \Z$.
Therefore, by the definition of divides, we know $5\mid c_{k+1} -2$.  \textit{[Inductive case done.]}







\end{enumerate}

\end{document}
