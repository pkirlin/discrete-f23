\documentclass[12pt, letterpaper]{report}
\usepackage{amsmath,fullpage,graphicx,amssymb}

\setlength{\parindent}{0em}

\newcommand{\nott}{{\sim}}
\newcommand{\Z}{\mathbb{Z}}
\newcommand{\Q}{\mathbb{Q}}

\newcommand{\indentpar}{\addtolength{\leftskip}{5mm}}
\newcommand{\stopindentpar}{\addtolength{\leftskip}{-5mm}}

\begin{document}

{\textbf{Discrete Structures, Fall 2017, Homework 6 Solutions}}

\medbreak

\textit{You must write the solutions to these problems legibly on your own paper, with
the problems in sequential order, and with all sheets stapled together.}

\bigskip

For each statement below, state whether it is true or false. Then prove the statement if it is true,
or its negation if it is false.\medskip

Remember, an example may only be used to prove that an existential statement is true or a universal
statement is false. Any example or counter-example must include specific values for the variables
and enough algebra and justification to illustrate that the example proves what you are claiming
it proves.\medskip

You do not need to translate each statement into symbols first, though it is often useful to do so.

\begin{enumerate}

        \item For any integers $a$ and $b$, if $a \mid b$, then $a \mid (a+b)$.  % true
        
        \textbf{Solution:} This is a true statement.

\textbf{Proof:}

Suppose $a$ and $b$ are arbitrarily chosen integers.

Assume $a \mid b$.

By the definition of divides, we know there exists an integer $k$ such that $ak=b$.

\textit{[We're trying to show $a \mid (a+b)$, so we want to get an equation that
 looks like  $a+b = a*(\text{an integer})$]}
 
$a + b = a + ak = a(1+k)$ by algebra and substitution.

$1+k$ is an integer by closure of the integers under addition.

Therefore, because $a+b = a(1+k)$ and $1+k \in \Z$, we know $a \mid (a+b)$ by the definition of divides.

        
        \item If $n$ is an odd integer, then $n^2-1$ is divisible by 4.  % true
        
        \textbf{Solution:} This is a true statement.

\textbf{Proof:}

Suppose $n$ is an arbitrary odd integer.  \textit{[Or, ``Suppose $n$ is an arbitrary integer
and assume $n$ is odd.''  Either one is fine; they mean the same thing.]}

By the definition of odd, there exists an integer $k$ such that $x=2k+1$.

\textit{[We're trying to show $n^2-1$ is divisible by 4, so we want to get an equation that
 looks like  $n^2-1=$ 4 * (an integer)]}

$n^2-1 = (2k+1)^2-1 = 4k^2 + 4k + 1 - 1 = 4k^2 + 4k$ by algebra.


$n^2-1 = (2k+1)^2-1 = 4k^2 + 4k = 4(k^2+k)$ by algebra.

Let $p=k^2+k$.  We know $p \in \Z$ because $\Z$ is closed under multiplication and addition. 

Therefore, $n^2-1 = 4p$, and therefore $4 \mid n^2-1$ by the definition of divides.

        
        \item $\forall a, b, c \in \Z \ [(a \mid c) \land (b \mid c)] \to [(a \mid b) \lor (b \mid a)]$.  % false
        
        \textbf{Solution:} This is a false statement.

\textbf{Counter-example:}  \textit{[Negation would be $\exists a, b, c \in \Z \ [(a \mid c) \land (b \mid c)] \land [(a \nmid b) \land (b \nmid a)]$.]}

Let $a=2$, $b=3$, and $c=6$.  Then it is true that $a\mid c$ and
$b \mid c$ (clearly $2 \mid 6$ and $3 \mid 6$), but it is not true that $a \mid b$ or $b \mid a$
(because 2 does not divide 3 and 3 does not divide 2).

        
        \item The product of any two consecutive integers is even.  
        
        Hints:  Use only one universally-quantified variable, not two.  Use the quotient-remainder theorem with $d=2$.
        
        
        
                        \textbf{Symbols:} $\forall x \in \Z \ \exists k \in \Z \ x(x+1)=2k$.

        \textbf{Solution:}

        Suppose $x$ is any arbitrary integer.

        By the quotient remainder theorem, there exists an integer $q$ such that $x=2q$ or $x=2q+1$.

        Case 1: Assume $x=2q$.

\indentpar
$x(x+1)=(2q)(2q+1) = 2[q(2q+1)]$ by algebra.

Choose $k=q(2q+1)$.  $k \in \Z$ by closure of the integers under addition and multiplication, and so therefore
we know $x(x+1)=2k$ for some integer $k$.

Therefore, $x(x+1)$ is even by the definition of even.

\stopindentpar
Case 2: Assume $x=2q+1$.

\indentpar
$x(x+1) = (2q+1)(2q+2) = (2q+1)(2)(q+1) = 2[(2q+1)(q+1)]$.

Choose $k=(2q+1)(q+1)$.  $k \in \Z$ by closure of the integers under addition and multiplication, and so therefore
we know $x(x+1)=2k$ for some integer $k$.

Therefore, $x(x+1)$ is even by the definition of even.

\stopindentpar


Because in both Case 1 and 2 we reach the same conclusion (that $x(x+1)$ is even),
and we know the quotient-remainder theorem tells us one of the two cases must be true, 
we can conclude that $x(x+1)$ must be even in general.
        
        
        
        \item For any integers $m$, $m^2-m$ can be written as either $3k$ or $3k+2$ for some integer $k$.
        
        Hints:  Use the quotient-remainder theorem on $m$.  See if you can determine on your own what $d$ should be.          

                
\textbf{Symbols: } $\forall m \in \Z \  [\exists k \in \Z \ m^2-m=3k \lor m^2-m=3k+2]$

\textbf{Proof:}

Let $m$ be an arbitrary integer.

By the quotient-remainder theorem, there exists an integer $q$ such that $m=3q$, $m=3q+1$,
or $m=3q+2$.

Case 1: Assume $m=3q$

\indentpar

$m^2-m = (3q)^2-3q = 9q^2 - 3q = 3(3q^2-q)$.

Let $k=3q^2-q$.  $k\in \Z$ by the closure of the integers under multiplication and addition.

So $m^2-m=3k$.

\stopindentpar

Case 2: Assume $m=3q+1$

\indentpar
\begin{align*}
m^2-m &= (3q+1)^2 - (3q+1) \\ 
&= 9q^2 + 6q + 1 -3q -1 \\&= 9q^2 +3q \\&= 3(3q^2+q)
\end{align*}
Let $k=3q^2+q$.  $k\in \Z$ by the closure of the integers under multiplication and addition.

So $m^2-m=3k$.

\stopindentpar

Case 3: Assume $m=3q+2$

\indentpar

\begin{align*}
m^2-m &= (3q+2)^2 - (3q+2) \\ 
&= 9q^2 + 12q + 4 -3q -2 \\&= 9q^2 +9q+2 \\&= 3(3q^2+3q)+2
\end{align*}

Let $k=3q^2+3q$.  $k\in \Z$ by the closure of the integers under multiplication and addition.

So $m^2-m=3k+2$.

\stopindentpar

In all three cases, we reach the conclusion that $m^2-m=3k$ or $m^2-m=3k+2$ for some integer
$k$.  Because we know one of the three cases must be true by the quotient-remainder 
theorem, we may conclude that $m^2-m=3k$ or $m^2-m=3k+2$ for some integer
$k$. 

\end{enumerate}
Hints: Remember, not all of these are necessarily true!  Use the step-by-step example proofs from the handouts last week
as a guide for these.
 
\bigskip
\textbf{Do the following problems about sequences and series.  Refer to section 5.1.}

\begin{enumerate}\setcounter{enumi}{5}
        
\item Write out the first four terms for each of the following sequences.  List the name
of the variable, the subscript, and the number itself.  For example, for ``$\forall n 
\in \Z^{\geq 1} \ d_n = 2n$'' you would write ``$d_1 = 2, \ d_2 = 4, \ d_3 = 6, \ d_4 = 8$.''

\begin{enumerate}
        \item $\forall i \in \Z^{\geq 2} \ a_i = i(i-1)$
        
        \textbf{Solution:} $a_2 = 2, a_3 = 6, a_4 = 12, a_5=20$.
        
        \item $\displaystyle \forall j \in \Z^{\geq 0} \ s_j = \frac{j}{j!}$
        
         \textbf{Solution:} $s_0 = 0, s_1 = 1, s_2 = 1, s_3 = 1/2$.
        
        \item $\displaystyle \forall k \in \Z^+ \ z_k = (1-k)(k-1)$
        
         \textbf{Solution:} $z_1 = 0, z_2 = -1, z_3 = -4, z_4 = -9$.
        
\end{enumerate}


\item Write the following sums using sigma ($\Sigma$) notation.

\begin{enumerate}

         \item $\displaystyle \frac{1}{2!} + \frac{2}{3!} + \frac{3}{4!} + \cdots + \frac{n}{(n+1)!}
        =\displaystyle \sum_{i=1}^n \frac{i}{(i+1)!}$
        \item $\displaystyle \frac{n}{1} + \frac{n-1}{2} + \frac{n-2}{3} + \cdots + \frac{1}{n}
        =\displaystyle \sum_{i=1}^n \frac{n-i+1}{i}$    
        \item $\displaystyle 1 - \frac{1}{2} + \frac{1}{3} - \frac{1}{4} + \cdots
        =\displaystyle \sum_{i=1}^\infty \frac{(-1)^{i+1}}{i}$\end{enumerate}

\item Rewrite each of the following summations as an equivalent expression by separating off the last term.
For reference, the first one has been done for you (we did this in class).

\begin{enumerate}

        \item (Example:) $\displaystyle \sum_{i=0}^{k+1}{i^2} = \sum_{i=0}^{k}{i^2} + (k+1)^2$
        \item $\displaystyle \sum_{i=1}^{m+1}{\frac{1}{2^{i-1}}}$ 
        
        \textbf{Solution:}
        $\displaystyle \sum_{i=0}^{m+1}{\frac{1}{2^{i-1}}} = \sum_{i=0}^{m}{\frac{1}{2^{i-1}}} + \frac{1}{2^m}$ 
        
        \item $\displaystyle \sum_{i=0}^{k}{\frac{i+1}{i+2}}$
        
         \textbf{Solution:} $\displaystyle \sum_{i=0}^{k}{\frac{i+1}{i+2}} =  \sum_{i=0}^{k-1}{\frac{i+1}{i+2}} + \frac{k+1}{k+2}$

\end{enumerate}

\end{enumerate}


\end{document}
