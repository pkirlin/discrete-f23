\documentclass[11pt, letterpaper]{report}
\usepackage{amsmath,fullpage,graphicx,amssymb}

\setlength{\parindent}{0em}

\newcommand{\nott}{{\sim}}

\newcommand{\Z}{\mathbb{Z}}
\newcommand{\R}{\mathbb{R}}

\newcommand{\powerset}[1]{\mathcal P \left({#1}\right)}

\begin{document}

{\textbf{Discrete Structures, Fall 2017, Homework 12 Solutions}}

\vspace*{.1in}

You must write the solutions to these problems legibly on your own paper, with
the problems in sequential order, and with all sheets stapled together.

\medskip

\textbf{For any problem that requires a numerical answer (as opposed to a proof or something written in words), unless otherwise specified, you do not need to fully reduce your
answer to a single number --- you may leave it in a form that uses addition, subtraction, multiplication, division, permutations (i.e., $P(n,k)$ notation)
and combinations (i.e., $\binom{n}{k}$ notation).}

\medskip

\textbf{Show your work for these problems! If you make a calculation error, it is easier to give partial credit if you illustrate how you derived your answer.}

\begin{enumerate}

\item Recall that in a standard deck of 52 playing cards, each card has both a \emph{suit}
and a \emph{rank}.  There are four suits, called \emph{clubs, diamonds, hearts,} and \emph{spades}.
There are thirteen ranks.  Nine of those ranks are named by the numbers 2 through 10.
The remaining four ranks are called \emph{jack, queen, king}, and \emph{ace}.  


In the poker variant of Texas Hold'em, there are multiple betting rounds.  In the first round, each player is dealt two playing cards at random 
from a standard deck of 52 cards.  For these problems, the order of the cards doesn't matter.
\begin{enumerate}
	     \item If you are dealt two cards at random from a standard deck of 52 cards, how many possible ways can this be done?  That is, how many possible two-card hands
        are there?
        
        \textbf{Solution:} $\binom{52}{2}$
        
        \item If you are dealt two cards at random from a standard deck of 52 cards, what is the probability your two cards are both face cards?  (Face cards are
        jacks, queens, and kings.  The two cards don't have to be the same face card.)
        
        
        \textbf{Solution:} There are four different suits, and each suit has three face cards,
        so there are 12 face cards per deck.
        
        The probability is $\displaystyle \frac{\displaystyle \binom{12}{2}}{\displaystyle \binom{52}{2}} = 11/221
        \approx 0.050$
        
        \item If you are dealt two cards at random from a standard deck of 52 cards, what is the probability you have a \emph{pocket pair}, meaning the two
        cards are of the same rank?  (e.g., both queens, both fives, both aces, etc)
        
                \textbf{Solution:}
        
        \emph{Method 1:}
                
        Pick the rank of the cards = $\binom{13}{1} = 13$ ways to do this.
        
        Pick two cards from the 4 of that suit = $\binom{4}{2}$ to do this.
        
        The probability is $\displaystyle \frac{\displaystyle \binom{13}{1}\binom{4}{2}}{\displaystyle \binom{52}{2}} = 1/17
        \approx 0.0588$
        
        \emph{Method 2:}
        
        Assume you have picked one card.  Calculate the probability that the second card matches
        the first card in rank.  There are 51 cards left to choose from (denominator), and 
        3 cards that match the first one in rank.
        
        The probability is 3/51 = 1/17.
        
        \emph{Method 3:}
        
        You can do this problem with order mattering, as long as you change both the numerator
        and denominator.  Note: this only works for probability problems; it would be incorrect
        to say the answer to part (a) is $52\cdot 51$, because order does not matter when
        just counting the hands.  When you do a probability problem, you can often introduce
        order in both the numerator and denominator because the extra factors will cancel each
        other out.
        
        Numerator: ways to pick first card = 52; ways to pick second card that matches in rank
         = 3.  So the numerator is $52 \cdot 3$.
         
        Denominator: ways to pick first card = 52; ways to pick second card = 51.  Denominator
        = $52\cdot 51$.
        
        The probability is $(52\cdot 3)/(52 \cdot 51) = 1/17.$
        
        \item If you are dealt two cards at random from a standard deck of 52 cards, what is the probability your hand is \emph{suited}, meaning the two cards
        are of the same suit?  (e.g., both diamonds, both clubs, etc)
        
                \textbf{Solution:}
        
        \emph{Method 1:}
                
        Pick the suit of the cards = $\binom{4}{1} = 4$ ways to do this.
        
        Pick two cards from the 13 of that suit = $\binom{13}{2}$ to do this.
        
        The probability is $\displaystyle \frac{\displaystyle \binom{4}{1}\binom{13}{2}}{\displaystyle \binom{52}{2}} = 4/17
        \approx 0.235$
        
        \emph{Method 2:}
        
        Assume you have picked one card.  Calculate the probability that the second card matches
        the first card in suit.  There are 51 cards left to choose from (denominator), and 
        12 cards that match the first one in rank.
        
        The probability is 12/51 = 4/17.
        
        \emph{Method 3:}
        
        Numerator: ways to pick first card = 52; ways to pick second card that matches in rank
         = 12.  So the numerator is $52 \cdot 12$.
         
        Denominator: ways to pick first card = 52; ways to pick second card = 51.  Denominator
        = $52\cdot 51$.
        
        The probability is $(52\cdot 12)/(52 \cdot 51) = 1/17.$
        
        \item If you are dealt two cards at random from a standard deck of 52 cards, what is the probability your two cards match in rank and suit?
        
        \textbf{Solution:}
        
        There is no way for the two cards to match in both rank and suit, because each card
        in a deck is unique and you drawing two cards at once; they must differ in either
        the rank or the suit.  The probability is zero.
        
        \item If you are dealt two cards at random from a standard deck of 52 cards, what is the probability your two cards have different ranks and different suits?
        
                \textbf{Solution:}
                
        \emph{Method 1:} 
        
        Numerator: ways to pick two cards without order with differing ranks and suits.
        First, pick two different ranks = $\binom{13}{2}$.  Second, pick a suit for the first
        rank = 4 ways.  Third, pick a suit for the second rank = 3 ways, since it cannot match
        the first suit.
        
        Numerator = $\binom{13}{2}\cdot 4 \cdot 3$.
        
        \emph{Note:} This is not the same as $\binom{13}{2}\binom{4}{2}$.  We can't use
        $\binom{4}{2}$ because that implies a hand of the king of hearts and the ace of spades
        is the same hand as the ace of hearts and the king of spaces.  In other words, once we
        pick the two ranks, we must use order to pick the suits = $P(4,2)=4\cdot 3$.
        
        Denominator = $\binom{52}{2}$.
        
        The probability is $\dfrac{\displaystyle \binom{13}{2}\cdot 4 \cdot 3}{\displaystyle \binom{52}{2}} = 12/17 = 0.706.$
        
        \emph{Method 2:} Use the difference rule.
        
        Another way to solve this is to start with all possible two-card hands, then remove
        those that match in rank, then remove those that match in suit.  (Note, this only works
        because there are no hands that match in rank and suit.  If there were, then the 
        procedure we just described would double-remove them.)
        
        Total hands = $\binom{52}{2}$.
        
        Matching ranks = $\binom{13}{1}\binom{4}{2}$.
        
        Matching suits = $\binom{4}{1}\binom{13}{2}$.
        
        The probability is $\dfrac{\binom{52}{2} - \binom{13}{1}\binom{4}{2} - \binom{4}{1}\binom{13}{2}}{\binom{52}{2}} = 12/17$.
        
        
        \end{enumerate}
	
	%\end{enumerate}




\item Rhodes is going to send a group of computer science majors to a local high school
to talk to the high schoolers about how cool CS is.  
\begin{enumerate}
        \item There are 20 CS majors.  How many ways can a group of 5 be picked to visit the 
        school?
        
           \textbf{Answer:} $\displaystyle \binom{20}{5} = \frac{20!}{5! \cdot 15!}$.

     
        
        \item The 20 CS majors consist of 12 first/second-year students and 8 third/fourth-year students.
        The group of 5 to visit the school should consist of at least one first/second-year student
        and at least one third/fourth-year student.  How many ways can the group be picked?
        
        Hint: Use the difference rule or the addition rule.
        
\textbf{Answer:}
        
        Method 1:
        
        Add up the number of ways to have 1 lowerclassman/4 upperclassmen, 2 lower/3 upper, 3/2,
        and 4/1:
        
        $\displaystyle \binom{12}{1}\binom{8}{4}
        +\binom{12}{2}\binom{8}{3} 
        +\binom{12}{3}\binom{8}{2}
        +\binom{12}{4}\binom{8}{1} \\= 
        \frac{12!}{1! \cdot 11!}\cdot\frac{8!}{4! \cdot 4!}+
        \frac{12!}{2! \cdot 10!}\cdot\frac{8!}{3! \cdot 5!}+
        \frac{12!}{3! \cdot 9!}\cdot\frac{8!}{2! \cdot 6!}+
        \frac{12!}{4! \cdot 8!}\cdot\frac{8!}{1! \cdot 7!}$
        
        Method 2:
        
        Figure out the total number of ways to choose any group (part A).  Now subtract
        out the ways to have all upper or all lower:
        
        $\displaystyle \binom{20}{5} - \binom{12}{5} - \binom{8}{5}$.
        
        Both of these methods yield the same result: 14656.
        
        \item A group of 5 is picked at random (not following the guidelines from part (b)).  What is the probability it consists of all first/second-years or all third/fourth years?
 
 
  \textbf{Answer:}
        
        $\displaystyle P(Event) = N(Event)/N(Sample\ space) = \dfrac{\displaystyle \binom{12}{5} + \binom{8}{5}}{\displaystyle \binom{20}{5}}
        = \dfrac{\dfrac{12!}{5! \cdot 7!} + \dfrac{8!}{5! \cdot 3!}}{\dfrac{20!}{5! \cdot 15!}}$.
        
        $=53/969 \approx 0.055$.
 



        \item Two other high schools get on board and want a group of 5 CS majors to visit.
        So now you need to pick 3 groups of 5 students each to send to the three schools.  How many ways can this be done?  (Class years don't matter for this problem.)
        
   

        Note that it matters which group goes to which school, but within each group, the ordering
        of the students doesn't matter. 
        
        Hint: Call the schools A, B, and C.  First, pick the students to visit school A.
        Then pick the students to visit school B.  Then pick the students to visit school C.
        
         \textbf{Answer:}
        
        $\displaystyle \binom{20}{5}\binom{15}{5}\binom{10}{5} = \dfrac{20!}{(5!)^4}=11,732,745,024$
        
        
        
\end{enumerate}


\item For this problem, assume Rhodes College has 2000 students.
\begin{enumerate}
	\item Is is guaranteed that among the Rhodes students, there are two students
	who share the same combination of initials of their first and last names?  (For instance,
	John Smith's initials are ``JS''.) Mathematically, explain why or why not.
	
	 \textbf{Answer:} Create two sets, $X$ = the set of all students, and $Y$ = the set of all combinations of
        first and last initials.  Size of $X$ = 2000.  Size of $Y$ = $26\cdot 26=676$.
        
        By the pigeonhole principle, any function from $X$ to $Y$ cannot be 1-1.  So when we assign
        students to their corresponding initials, there must be two students who are mapped to the
        same initials.  In other words, (at least) two students on campus \emph{must} have the same initials.
        

	
	
	
	\item Is it guaranteed that among the Rhodes students, there are three students
	who share the same initials?  How about four students?  Mathematically, explain why or why not for each
	case.
	
	
	\textbf{Answer:} Assume we have the same sets $X$ and $Y$ as in the previous part.
        
        By the pigeonhole principle, if the inequality $n > k\cdot m$ is true, with the size of set $X$
        being $n$ and the size of $Y$ being $m$, then there must be $k+1$
        pigeons in $X$ that end up in a single pigeonhole in $Y$.  So if we want to know if there are three
        students with the same initials, we use $k=2$.  The inequality $2000 > 2 \cdot 676$ is true, so 
        there \emph{are} three students who share the same initials.
        
        The case for 4 is not true however.  The inequality $2000 > 3\cdot 676$ is false (because $3\cdot 676=2028$) and so we cannot say (for sure) that four students share initials.  There is a possibility
        of this happening, but the pigeonhole principle can only be used for a guarantee, and we don't
        meet the conditions to have the guarantee.
        

	
	
	\vspace{.1in}
	
	 \emph{The next three questions are all related.}
	 
	 \vspace{.1in}
	
	\item Suppose the Rat owns owns 500 forks and 500 spoons.  Every time a student comes into the Rat to eat a meal, they always select (at random) one fork and one spoon to eat
	with.  
	
	How many possible ways can a fork-and-spoon pair be chosen?  (Ignore any issues of some forks or spoons being unavailable because other students are already using them.)
	
	\textbf{Answer:}  $500 \cdot 500=250,000$ ways.
	
	\item By the pigeonhole principle, how many meals must the Rat serve to guarantee that the exact same fork-and-spoon pair was chosen twice?  (Assume ``meal'' means one student eating
a meal, not all students eating a meal.  This question is not supposed to be tricky.)

\textbf{Answer:} The Rat must serve at least $250,001$ meals.  Any smaller than that, and there's a chance that every student picked up a different fork/spoon pairing.  Once you get
above $250,000$, there must be a pair that was used twice.
	
	\item Now assume that all 2000 students eat every meal in the Rat.  (That is, each student eats three meals per day there.)  
	
	What is the minimum number of days that the Rat has to serve meals
	to guarantee that the same fork-and-spoon pair was chosen twice? 	\emph{Your answer should be an integer.}
	Explain mathematically.
	
	\textbf{Answer:} We must figure out on what day the $250,001$st meal was served.  $2000$ students eat three meals per day; that is $6000$ meals served every day.  After day 1,
	6000 meals have been served; after day 2, 12000 meals, and so on.  If we divide $250,0001$ by $6000$, we come up with a number between 41 and 42.  Therefore, some time during
	day 42, the $250,001$st meal will be served.  So the answer is 42 days.
	
\end{enumerate}
\item Suppose I pick three integers arbitrarily.  Use the pigeonhole principle to explain why among those three
integers,
there must be a pair of integers whose difference is even.  (You may state whatever
facts you want about even or odd numbers without proof, as long as your statements
are true.)

\textbf{Solution:}  Among any set of three integers, either two of them must be even, or
two of them must be odd.  This can be determined from the pigeonhole principle: there
are three integers and two properties (even and odd), so any function that maps the integers
to even/odd must assign two of them to the same property.  

So now we now two of the integers are even or two are odd.  We also know that the difference
of two even integers is even, and the difference of two odd integers is also even (we state
these facts without proof, though they are straightforward to prove).  Therefore, in 
either case, there is a pair of integers whose difference is even.






\end{enumerate}
\end{document}
