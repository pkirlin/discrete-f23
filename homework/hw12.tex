\documentclass[11pt, letterpaper]{report}
\usepackage{amsmath,fullpage,graphicx,amssymb}

\setlength{\parindent}{0em}

\newcommand{\nott}{{\sim}}

\newcommand{\Z}{\mathbb{Z}}
\newcommand{\R}{\mathbb{R}}

\newcommand{\powerset}[1]{\mathcal P \left({#1}\right)}

\begin{document}

{\textbf{Discrete Structures, Fall 2017, Homework 12}}

\vspace*{.1in}

You must write the solutions to these problems legibly on your own paper, with
the problems in sequential order, and with all sheets stapled together.

\medskip

\textbf{For any problem that requires a numerical answer (as opposed to a proof or something written in words), unless otherwise specified, you do not need to fully reduce your
answer to a single number --- you may leave it in a form that uses addition, subtraction, multiplication, division, permutations (i.e., $P(n,k)$ notation)
and combinations (i.e., $\binom{n}{k}$ notation).}

\medskip

\textbf{Show your work for these problems! If you make a calculation error, it is easier to give partial credit if you illustrate how you derived your answer.}

\begin{enumerate}

\item Recall that in a standard deck of 52 playing cards, each card has both a \emph{suit}
and a \emph{rank}.  There are four suits, called \emph{clubs, diamonds, hearts,} and \emph{spades}.
There are thirteen ranks.  Nine of those ranks are named by the numbers 2 through 10.
The remaining four ranks are called \emph{jack, queen, king}, and \emph{ace}.  


In the poker variant of Texas Hold'em, there are multiple betting rounds.  In the first round, each player is dealt two playing cards at random 
from a standard deck of 52 cards.  For these problems, the order of the cards doesn't matter.
\begin{enumerate}
	\item If you are dealt two cards at random from a standard deck of 52 cards, how many possible ways can this be done?  That is, how many possible two-card hands
	are there?
	\item If you are dealt two cards at random from a standard deck of 52 cards, what is the probability your two cards are both face cards?  (Face cards are
	jacks, queens, and kings.  The two cards don't have to be the same face card.)
	\item If you are dealt two cards at random from a standard deck of 52 cards, what is the probability you have a \emph{pocket pair}, meaning the two
	cards are of the same rank?  (e.g., two queens, two fives, two aces, etc)
	\item If you are dealt two cards at random from a standard deck of 52 cards, what is the probability your hand is \emph{suited}, meaning the two cards
	are of the same suit?  (e.g., two diamonds, two clubs, etc)
	\item If you are dealt two cards at random from a standard deck of 52 cards, what is the probability your two cards match in rank \emph{and} suit?
	\item If you are dealt two cards at random from a standard deck of 52 cards, what is the probability your two cards have different ranks \emph{and} different suits?
	
	
	
	\end{enumerate}




\item Rhodes is going to send a group of computer science majors to a local high school
to talk to the high schoolers about how cool CS is.  
\begin{enumerate}
        \item There are 20 CS majors.  How many ways can a group of 5 be picked to visit the 
        school?
        
        
        
        \item The 20 CS majors consist of 12 first/second-year students and 8 third/fourth-year students.
        The group of 5 to visit the school should consist of at least one first/second-year student
        and at least one third/fourth-year student.  How many ways can the group be picked?
        
        Hint: Use the difference rule or the addition rule.
        

        
        \item A group of 5 is picked at random (not following the guidelines from part (b)).  What is the probability it consists of all first/second-years or all third/fourth years?
 

        \item Two other high schools get on board and want a group of 5 CS majors to visit.
        So now you need to pick 3 groups of 5 students each to send to the three schools.  How many ways can this be done?  (Class years don't matter for this problem.)

        Note that it matters which group goes to which school, but within each group, the ordering
        of the students doesn't matter. 
        
        Hint: Call the schools A, B, and C.  First, pick the students to visit school A.
        Then pick the students to visit school B.  Then pick the students to visit school C.
\end{enumerate}


\item For this problem, assume Rhodes College has 2000 students.
\begin{enumerate}
	\item Is is guaranteed that among the Rhodes students, there are two students
	who share the same combination of initials of their first and last names?  (For instance,
	John Smith's initials are ``JS''.) Mathematically, explain why or why not.
	\item Is it guaranteed that among the Rhodes students, there are three students
	who share the same initials?  How about four students?  Mathematically, explain why or why not for each
	case.
	\vspace{.1in}
	
	 \emph{The next three questions are all related.}
	 
	 \vspace{.1in}
	
	\item Suppose the Rat owns owns 500 forks and 500 spoons.  Every time a student comes into the Rat to eat a meal, they always select (at random) one fork and one spoon to eat
	with.  
	
	How many possible ways can a fork-and-spoon pair be chosen?  (Ignore any issues of some forks or spoons being unavailable because other students are already using them.)
	
	\item By the pigeonhole principle, how many meals must the Rat serve to guarantee that the exact same fork-and-spoon pair was chosen twice?  (Assume ``meal'' means one student eating
a meal, not all students eating a meal.  This question is not supposed to be tricky.)
	
	\item Now assume that all 2000 students eat every meal in the Rat.  (That is, each student eats three meals per day there.)  
	
	What is the minimum number of days that the Rat has to serve meals
	to guarantee that the same fork-and-spoon pair was chosen twice? 	\emph{Your answer should be an integer.}
	Explain mathematically.
\end{enumerate}
\item Suppose I pick three integers arbitrarily.  Use the pigeonhole principle to explain why among those three
integers,
there must be a pair of integers whose difference is even.  (You may state whatever
facts you want about even or odd numbers without proof, as long as your statements
are true.)





\end{enumerate}
\end{document}
