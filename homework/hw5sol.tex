\documentclass[12pt, letterpaper]{report}
\usepackage{amsmath,fullpage,graphicx,amssymb}

\setlength{\parindent}{0em}

\newcommand{\nott}{{\sim}}
\newcommand{\Z}{\mathbb{Z}}
\newcommand{\Q}{\mathbb{Q}}

\begin{document}

{\textbf{Discrete Structures, Fall 2017, Homework 5 Solutions}}

\medbreak

\textit{You must write the solutions to these problems legibly on your own paper, with
the problems in sequential order, and with all sheets stapled together.}

\bigskip

For each statement below, state whether it is true or false. Then prove the statement if it is true,
or its negation if it is false.\medskip

Remember, an example may only be used to prove that an existential statement is true or a universal
statement is false. Any example or counter-example must include specific values for the variables
and enough algebra and justification to illustrate that the example proves what you are claiming
it proves.\medskip

You do not need to translate each statement into symbols first, though it is often useful to do so.

\begin{enumerate}

        \item The product of any two odd integers is odd.  % true
        
        
        \textbf{Symbols:} $\forall a, b \in \Z^{\text{odd}} \ \text{Odd}(a\cdot b)$
        
        \textbf{This statement is true.}
        
        \textbf{Proof:}
        
        Suppose $a$ and $b$ are two arbitrarily chosen odd integers.
        
        By the definition of odd, there exist integers $k$ and $p$ such that $a=2k+1$ and $b=2p+1$.
        
        $a \cdot b = (2k+1)(2p+1) = 4kp + 2p + 2k + 1 = 2(2kp + p + k) + 1$ by substitution and algebra.
        
        Let $s = 2kp + p + k$.  $s \in \Z$ by closure of the integers under multiplication and addition.
        
        Because $a\cdot b = 2s+1$, we can say that $a\cdot b$ is odd by the definition of odd.
        

        
        \item For any integers $a$ and $b$, if $a-b$ is even, then $a$ and $b$ are both even.
        
        \textbf{Symbols:} $\forall a,b \in \Z \ \text{Even}(a-b) \to (\text{Even}(a)\land \text{Even}(b))$
 
        
        \textbf{This statement is false.}
        
        \textbf{Counterexample}: Let $a=5$ and $b=3$.  Then $a-b=5-3=2$.  2 is clearly even, but $a$ and $b$ are odd.
        
        \item If $n$ is an even integer, then $n^2-2$ is even.  % true
        
        \textbf{Symbols:} $\forall n \in \Z \ \text{Even}(n) \to \text{Even}(n^2-2)$
        
        \textbf{This statement is true.}
        
        \textbf{Proof:}
        
        Suppose $n$ is an arbitrarily-chosen integer.  
        
        Assume $n$ is even.
        
        By the definition of even, there exists an integer $k$ such that $n=2k$.
        
        $n^2-2 = (2k)^2-2 = 4k^2-2 = 2(2k^2-1)$ by algebra/substitution.
        
        Let $m = 2k^2-1$.  $m$ is an integer by closure of the integers under multiplication and addition.
        
        Therefore, $n^2-2 = 2m$, and so $n^2-2$ is even by the definition of even.
        
        \item The sum of an integer and a rational number is rational.  % true
        
          
        
    
        

        
        \textbf{Symbols:} $\forall n \in \Z \ \ \forall r \in \Q \ \ \text{Rational}(n+r)$
        
        \textbf{This statement is true.}
        
        \textbf{Proof:}
        
        Suppose $n$ is an arbitrarily-chosen integer, and that $r$ is an arbitrarily-chosen rational number.
        
        By the definition of rational, there exist integers $a$ and $b$ such that $r=a/b$ and $b\neq 0$.
        
        $\displaystyle n+r = n + \frac{a}{b} = \frac{nb+a}{b}  \text{ \ \ by substitution and algebra.}$
        
        Let $p = nb+a$.  
        
        $p$ is an integer because the integers are closed under multiplication and addition.
                
        Because $n+r = p/b$, and we know $p \in \Z$, $b \in \Z$, and $b \neq 0$, we can say that  $n+r$ is
        rational by the definition of rational.


        
       	\item If $n$ and $m$ are rational numbers, then $n/m$ is a rational number.
	
	\textbf{Symbols:} $\forall n,m \in \Q \ \ \text{Rational}(n/m)$

            \textbf{This statement is false.}
        
        Counterexample: Let $n=2$ and $m=0$.  $n$ is a rational number because it can be written as $2/1$, and $m$ is rational
        because it can be written as $0/1$.  \emph{[Alternatively, you can just say $n$ and $m$ are rational because they are integers
        and all integers are rational.]}
        
        $n/m = 2/0$ which is not rational because the denominator is zero (and therefore $n/m$ is undefined).
                

\end{enumerate}
\end{document}
