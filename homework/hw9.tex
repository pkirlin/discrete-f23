\documentclass[11pt, letterpaper]{report}
\usepackage{amsmath,fullpage,graphicx,amssymb}

\setlength{\parindent}{0em}

\newcommand{\nott}{{\sim}}

\newcommand{\Z}{\mathbb{Z}}

\newcommand{\powerset}[1]{\mathcal P\left({#1}\right)}

\newcommand{\proofnote}[1]{[\textit{Note: #1}]}

\begin{document}

\textbf{Discrete Structures, Fall 2023, Homework 9}

\bigskip

You must write the solutions to these problems legibly on your own paper, with
the problems in sequential order, and with all sheets stapled together.

\begin{enumerate}

	\item Prove the following statement using an \textit{element proof}:
	
	For any sets $A$, $B$, $C$, and $D$, if $C \subseteq (D \cup A)$ and $B \subseteq D^c$, then
	$C \cap B \subseteq A$.
	
	\item Prove the following statement using an \textit{element proof}:
	
	For any sets $A$, $B$, and $C$, if $A \subseteq C$ and $B \subseteq C$, then
	$A \cup B \subseteq C$.
	
	Hint: You will need to divide the proof into two cases at some point, like in practice problem 2 below.
	
\end{enumerate}

SET PRACTICE PROBLEMS: (\textit{These are not part of the homework; solutions are on the next page.})

\begin{enumerate}

	\item Prove the following statement using an \textit{element proof}:
	
	For any sets $A$, $B$, and $C$, if $A \subseteq B$ and $B \subseteq C$, then
	$A \subseteq C$.
	
	\item Prove the following statement using an \textit{element proof}:
	
	For any sets $A$, $B$, and $C$, if $A \subseteq B$, then
	$(A \cup C) \subseteq (B \cup C)$.
	
	\item Prove the following statement using an \textit{element proof}:
	
	For any sets $A$, $B$, $C$, $D$, and $E$, if $A \subseteq (B \cup C)^c$ and $D \subseteq E$, then
	$(A \cap D) \subseteq (E-B)$.
	
\end{enumerate}


\newpage
SOLUTIONS TO PRACTICE PROBLEMS:



\begin{enumerate}

	\item Prove the following statement using an \textit{element proof}:
	
	For any sets $A$, $B$, and $C$, if $A \subseteq B$ and $B \subseteq C$, then
	$A \subseteq C$.
	
	\textbf{Proof:}
	
	Let $A$, $B$, and $C$ be arbitrary sets.
        
        Assume $A \subseteq B$ and $B \subseteq C$.
        
        \proofnote{We are trying to prove that $A \subseteq C$, so let's assume $x$ is an arbitrary element in the left side of the subset
        ($A$) and show $x$ must be in the right side ($C$).}
        
        Let $x$ be an arbitrary element in $U$.  Assume $x \in A$.  
        
        \proofnote{The line above could also be written as ``Let $x$ be an arbitrary element in $A$.''}
        
        $x \in A$ \qquad (from above)
        
        $x \in B$ \qquad (because $A \subseteq B$ and $x \in A$)
        
        $x \in C$ \qquad (because $B \subseteq C$ and $x \in B$)
        
                
        Because we assumed $x$ was an arbitrary element in $A$ and we showed $x \in C$, we can conclude (by the definition of subset)
        that $A \subseteq C$.

	
	\item Prove the following statement using an \textit{element proof}:
	
	For any sets $A$, $B$, and $C$, if $A \subseteq B$, then
	$(A \cup C) \subseteq (B \cup C)$.
	
	\textbf{Proof:}
	
	Let $A$, $B$, and $C$ be arbitrary sets.
        
        Assume $A \subseteq B$.
        
        \proofnote{We are trying to prove that $(A \cup C) \subseteq (B \cup C)$, so let's assume $x$ is an arbitrary element in the left side of the subset
        ($A \cup C$) and show $x$ must be in the right side ($B \cup C$).}
        
        Let $x$ be an arbitrary element in $U$.  Assume $x \in A \cup C$.  
        
        \proofnote{The line above could also be written as ``Let $x$ be an arbitrary element in $A \cup C$.''}
        
        $x \in A \cup C$ \qquad (from above)
        
        $x \in A \lor  x \in C$ \qquad (def of union)
        
        \proofnote{Because we don't know if $x \in A$ or if $x \in C$, we will split the proof into cases,
        one case where we assume $x \in A$ and one where we assume $x \in C$.  We will make sure both cases
        lead to the same conclusion, that $x \in B \cup C$.  And then we will use the rule of dilemma/proof by
        division into cases to conclude that whichever side of the ``or'' statement above is true 
        ($x \in A \lor  x \in C$), since both sides lead to the same conclusion (that $x \in B \cup C$),
        that the conclusion must be true in general.}
        
        \textbf{Case 1:} Assume $x \in A$.
        
        \quad $x \in B$ \qquad (because $x\in A$ and $A \subseteq B$)
        
        \quad $x \in B \lor x \in C$ \qquad (disjunctive addition)	
        
        \quad $x \in B \cup C$ \qquad (def of union)	
        
        \textbf{Case 2:} Assume $x \in C$.
       
        
        \quad $x \in B \lor x \in C$ \qquad (disjunctive addition)	
        
        \quad $x \in B \cup C$ \qquad (def of union)	
        
        Since both cases above lead to the same conclusion, we can conclude that $x \in B \cup C$.
        
         Because we assumed $x$ was an arbitrary element in $A \cup C$ and we showed $x \in B \cup C$, we can conclude (by the definition of subset)
        that $A \cup C \subseteq B \cup C$.
        
	\item Prove the following statement using an \textit{element proof}:
	
	For any sets $A$, $B$, $C$, $D$, and $E$, if $A \subseteq (B \cup C)^c$ and $D \subseteq E$, then
	$(A \cap D) \subseteq (E-B)$.
	
	\textbf{Proof:}
	
	Let $A$, $B$, $C$, $D$, and $E$ be arbitrary sets.
        
        Assume $A \subseteq (B \cup C)^c$ and $D \subseteq E$.
        
        \proofnote{We are trying to prove that $(A \cap D) \subseteq (E-B)$, so let's assume $x$ is an arbitrary element in the left side of the subset
        ($A \cap D$) and show $x$ must be in the right side ($E-B$).}
        
        Let $x$ be an arbitrary element in $U$.  Assume $x \in A \cap D$.  
        
        \proofnote{The line above could also be written as ``Let $x$ be an arbitrary element in $A\cap D$.''}

	$x \in A \cap D$ \qquad (from above)
        
    $x \in A \land  x \in D$ \qquad (def of intersection)
    
    $x \in A$ \qquad (conjunctive simplification)
    
    $x \in D$ \qquad (conjunctive simplification)
    
    $x \in (B \cup C)^c$ \qquad (because $x \in A$ and $A \subseteq (B \cup C)^c$)
    
    $\nott (x \in (B \cup C))$ \qquad (def of complement)
    
    $\nott (x \in B \lor x \in C)$ \qquad (def of union)
    
    $\nott (x \in B) \land \nott( x \in C)$ \qquad (deMorgan's law)
    
    $\nott (x \in B)$ \qquad (conjunctive simplification)
    
    $x \in B^c$ \qquad (def of complement)
    
    \proofnote{We also could have turned $x \in (B \cup C)^c$ into $x \in (B^c \cap C^c)$ through the set version of deMorgan's laws, and proceeded from there.  We would still want to get $x \in B^c$.}
    
    $x \in E$ \qquad (because $x \in D$ and $D \subseteq E$)
    
    $x \in E \land x \in B^c$ \qquad (conjunctive addition)
    
    $x \in E \cap B^c$ \qquad (def of intersection)
    
    $x \in E - B$ \qquad (def of set difference)
    
    Because we assumed $x$ was an arbitrary element in $A \cap D$ and we showed $x \in B -E$, we can conclude (by the definition of subset)
        that $A \cap D \subseteq E-B$.

	
\end{enumerate}

\end{document}
