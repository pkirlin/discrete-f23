\documentclass[12pt, letterpaper]{report}
\usepackage{amsmath,fullpage,graphicx,amssymb}

\setlength{\parindent}{0em}

\newcommand{\nott}{{\sim}}
\newcommand{\Z}{\mathbb{Z}}

\begin{document}

{\textbf{Discrete Structures, Fall 2017, Homework 6}}

\medbreak

\textit{You must write the solutions to these problems legibly on your own paper, with
the problems in sequential order, and with all sheets stapled together.}

\bigskip

For each statement below, state whether it is true or false. Then prove the statement if it is true,
or its negation if it is false.\medskip

Remember, an example may only be used to prove that an existential statement is true or a universal
statement is false. Any example or counter-example must include specific values for the variables
and enough algebra and justification to illustrate that the example proves what you are claiming
it proves.\medskip

You do not need to translate each statement into symbols first, though it is often useful to do so.

\begin{enumerate}

        \item For any integers $a$ and $b$, if $a \mid b$, then $a \mid (a+b)$.  % true
        
        \item If $n$ is an odd integer, then $n^2-1$ is divisible by 4.  % true
        
        \item $\forall a, b, c \in \Z \ [(a \mid c) \land (b \mid c)] \to [(a \mid b) \lor (b \mid a)]$.  % false
        
        \item The product of any two consecutive integers is even.  
        
        Hints:  Use only one universally-quantified variable, not two.  Use the quotient-remainder theorem with $d=2$.
        
        
        
        \item For any integers $m$, $m^2-m$ can be written as either $3k$ or $3k+2$ for some integer $k$.
        
        Hints:  Use the quotient-remainder theorem on either $m$.  See if you can determine on your own what $d$ should be.          

                

\end{enumerate}
Hints: Remember, not all of these are necessarily true!  Use the step-by-step example proofs from the handouts last week
as a guide for these.
 
\bigskip
\textbf{Do the following problems about sequences and series.  Refer to section 5.1.}

\begin{enumerate}\setcounter{enumi}{5}
        
\item Write out the first four terms for each of the following sequences.  List the name
of the variable, the subscript, and the number itself.  For example, for ``$\forall n 
\in \Z^{\geq 1} \ d_n = 2n$'' you would write ``$d_1 = 2, \ d_2 = 4, \ d_3 = 6, \ d_4 = 8$.''

\begin{enumerate}
        \item $\forall i \in \Z^{\geq 2} \ a_i = i(i-1)$
        \item $\displaystyle \forall j \in \Z^{\geq 0} \ s_j = \frac{j}{j!}$
        \item $\displaystyle \forall k \in \Z^+ \ z_k = (1-k)(k-1)$
\end{enumerate}

\item Write the following sums using sigma ($\Sigma$) notation.

\begin{enumerate}

        \item $\displaystyle \frac{1}{2!} + \frac{2}{3!} + \frac{3}{4!} + \cdots + \frac{n}{(n+1)!}$
        \item $\displaystyle \frac{n}{1} + \frac{n-1}{2} + \frac{n-2}{3} + \cdots + \frac{1}{n}$        
        \item $\displaystyle 1 - \frac{1}{2} + \frac{1}{3} - \frac{1}{4} + \cdots$

\end{enumerate}

\item Rewrite each of the following summations as an equivalent expression by separating off the last term.
For reference, the first one has been done for you (we did this in class).

\begin{enumerate}

        \item (Example:) $\displaystyle \sum_{i=0}^{k+1}{i^2} = \sum_{i=0}^{k}{i^2} + (k+1)^2$
        \item $\displaystyle \sum_{i=1}^{m+1}{\frac{1}{2^{i-1}}}$ 
        \item $\displaystyle \sum_{i=0}^{k}{\frac{i+1}{i+2}}$

\end{enumerate}

\end{enumerate}


\end{document}
